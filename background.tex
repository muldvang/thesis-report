% chktex-file 24
% chktex-file 44

\chapter{Background}
\label{cha:background}

\fxnote{Meta text.}

Hidden Markov models, from now on abbreviated HMMs, were first introduced and
studied in the end of the 1960s and the beginning of the 1970s. Since then they
have become very popular and have proven to be both effective and efficient in
many applications. \fxnote{Move this to the section below?}

\section{Hidden Markov Models}
\label{sec:hidden-markov-models}

Think of a system that can be described as a being in one of $N$ states and any
time. Let the set states be notated as $\mathcal{H} = {h_1, h_2, \dots, h_N}$.
In figure~\ref{fig:markov-chain} an example of a with three states is shown. In
some discrete time steps denoted $t = 1, 2, \dots$ the system changes state to
a new state or back to the same state. For each state there is a certain
probability of making a transition to any other state. This may be represented
as a matrix $A = {\{a_{ij}\}}_{1 \le i \le N}$ with $a_{ij} \in [0, 1]$ and
$\sum_{j = 1}^N a_{ij} = 1$. This is called a Markov process, or more precisely
for discrete-time and finite state spaces: a Markov chain. The non-zero
transition probabilities is shown as arrows in
figure~\ref{fig:markov-chain}. The initial state, at time $t = 1$, is
determined from a vector $\Pi = {(\pi_i)}_{1 \le i \le N}$ with
$\sum_{i=1}^N \pi_i = 1$ being the probability of starting in each state.

\begin{figure}
  \centering
  Figure here!
  \caption{An example of a Markov chain.}
  \label{fig:markov-chain}
\end{figure}

Having established the above system with $A$ and $\Pi$ given, one may ask
questions like what the probability of observing a certain sequence of states
$X_{1:T}$ during time steps $t = 1, \dots\ T$ is. As an example lets assume the
model in figure~\ref{fig:markov-chain} is a model of the weather with $h_1$
being rainy, $h_2$ being cloudy, and $h_3$ being sunny weather. The probability
of a sequence of observations $X_{1:4} = x_1x_2x_3x_4$ with
$x_i \in \mathcal{H}$ over four days, $t = 1, 2, 3, 4$, may easily be computed
by multiplying probabilities.

The above formality is too limited for many applications, since the states do
not correspond to the actual observable events. To overcome this problem a
distinction between observables and states is made. The observables is made a
probabilistic function of the state. Given
$\mathcal{O} = {o_1, o_2, \dots, o_M}$ being a set of $M$ observables an
emission matrix $B = {\{b_{ij}\}}_{1 \le i \le N}^{1 \le j \le M}$ with
$\sum_{j=1}^M b_{ij} = 1$ is defined, where $b_{ij}$ is the probability of
observing $o_j$ when the system is in state $h_i$. The set of states,
$\mathcal{H}$, may now be called ``hidden'' states since they are not
observable. An example of an HMM is shown in
figure~\ref{fig:hidden-markov-model}. The figure is similar to
figure~\ref{fig:markov-chain} except for the probabilities added for each
hidden state.

\begin{figure}
  \centering
  More figure here!
  \caption{An example of a hidden Markov model.}
  \label{fig:hidden-markov-model}
\end{figure}

For modelling weather we can now use the fact that the observables (rainy,
cloudy, and sunny weather) depend on the air preasure. This may be modeled as
$h_1$ being low preasure, $h_2$ being high preasure, $o_1$ being rainy, $o_2$
being cloudy, and $o_3$ being sunny weather. As before the probability of a
sequence of observations $Y_{1:4} = y_1y_2y_3y_4$ with $y_i \in \mathcal{O}$
can be found.

Be problem scetched above of finding to joint probability of a sequence of
observations $Y_{1:T}$ is solved by the Forward algorithm. Another problem
addressed in this thesis is to find the \emph{most likely} sequence of hidden
states $X_{1:T}^* = x_1^*, x_2^*, \dots, x_T^*$ emitting $Y_{1:T}$. This is
solved by the Viterbi algorithm. Finally, the posterior decoding algorithm
solves the problem of finding the most likely state to be in at each timestep
$t$. This is similar to the Viterbi decoding except that the sequence of hidden
states may not be valid according to the model.

\section{Related Work}

\citet{lifshits2009speeding} present a method for speeding up the dynamic
programming algorithms used with HMMs, namely the forward-backward algorithms
and the Viterbi algorithm. The approach is based on finding repeated substrings
in the observed input sequence. These substrings are found using five different
algorithms: Four Russians method, run length encoding, Lempel-Ziv parsing,
grammer-based compression and byte pair encoding. The forward-backward and the
Viterbi algorithms are the reformulated into series of matrix
multiplications. The overall idea is that the repeated substrings correspond to
repeated matrix multiplications and by finding the repeated substrings the
multiplications can be avoided. The article is primarily theoretical. A single
experiment have been performed on some DNA sequences showing a speed up of the
Viterbi algorithm when using an improved Lempel-Ziv parsing without
backtracking. No experiments have been made for the other algorithms and no
code has been made available.

\citet{sand2013ziphmmlib} present zipHMMlib, a highly optimized HMM library for
speeding up the forward algorithm. The speed up is achieved by finding repeated
substrings using byte pair encoding. Much of the theory in this paper rely on
\cite{lifshits2009speeding}, but it is extended to make the computations
numerically stable. Furthermore, the code is available as an open source
library with bindings for both Python and R.

This thesis extends the work made by \citet{sand2013ziphmmlib} to also include
a highly efficient Viterbi algorithm based on byte pair encoding as developed
theoretically by \citet{lifshits2009speeding}. The theory will also be extended
a bit to make the computation numerically stable and the experiments will be
more extensive.

%%% Local Variables:
%%% mode: latex
%%% TeX-master: "master"
%%% End:
