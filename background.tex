% chktex-file 24
% chktex-file 44

\chapter{Background}
\label{cha:background}

In this chapter the preliminary theory of hidden Markov models, and
the work by \citet{lifshits2009speeding} and \citet{sand2013ziphmmlib} is
described.

\section{Hidden Markov Models}
\label{sec:hidden-markov-models}

Hidden Markov models (HMMs) were first introduced and
studied in the end of the 1960s and the beginning of the 1970s. Since then they
have become very popular and have proven to be both effective and efficient in
many applications \citep{rabiner1989tutorial}.

\begin{figure}
  \centering
  \tikzsetnextfilename{markov_chain}
\begin{tikzpicture} [
  ->,
  >=stealth',
  auto,
  node distance=2cm,
  main node/.style={rectangle,draw,rounded corners},
  ]

  \node[main node] (h1) {$h_1$};
  \node[main node, below of=h1, right of=h1] (h2) {$h_2$};
  \node[main node, below of=h2, left of=h2] (h3) {$h_3$};
  \node[left of=h1] (p1) {};
  \node[right of=h2] (p2) {};
  \node[left of=h3] (p3) {};

  \path
    (h1) edge[bend right]               node[right] {$a_{12}$} (h2)
         edge                           node[left]  {$a_{13}$} (h3)
    (h2) edge[bend right]               node[right] {$a_{21}$} (h1)
         edge[bend right]               node        {$a_{23}$} (h3)
    (h3) edge[bend right]               node[right] {$a_{32}$} (h2)
         edge[loop below, looseness=20] node        {$a_{33}$} (h3);

  \path
    (p1) edge node {$\pi_1$} (h1)
    (p2) edge node {$\pi_2$} (h2)
    (p3) edge node {$\pi_3$} (h3);
\end{tikzpicture}

%%% Local Variables:
%%% mode: latex
%%% TeX-master: "../master"
%%% End:

  \caption{An example of a Markov chain.}
  \label{fig:markov-chain}
\end{figure}

Think of a system that can be described as a being in one of $N$ states at any
time. Let the set of states be notated as $\mathcal{H} = \{h_1, h_2, \dots, h_N\}$.
In figure~\ref{fig:markov-chain} an example with three states is shown. In
discrete time steps denoted $t = 1, 2, \dots$ the system changes from one state
to another. For each state there is a certain probability
of making a transition to any other state. This can be represented as a matrix
$A = {\{a_{ij}\}}_{1 \le i \le N}$ with $a_{ij} \in [0, 1]$ and
$\sum_{j = 1}^N a_{ij} = 1$. A system like this where the next state is only
dependent on the current is called a Markov process, or more precisely
for discrete-time and finite state spaces: a Markov chain. The non-zero
transition probabilities are shown as arrows in
figure~\ref{fig:markov-chain}. The initial state, at time $t = 1$, is
determined from a vector $\Pi = {(\pi_i)}_{1 \le i \le N}$ where
$\sum_{i=1}^N \pi_i = 1$ with $\pi_i$ being the probability of starting in
state $\pi_i$.

Having established the above system with $A$ and $\Pi$ given, one might ask
questions like what the probability of observing a certain sequence of states
$V_{1:T}$ during time steps $t = 1, \dots\ T$ is. As an example lets assume the
model in figure~\ref{fig:markov-chain} is a model of the weather with $h_1$
being rainy, $h_2$ being cloudy, and $h_3$ being sunny weather. The probability
of a sequence of observations $V_{1:4} = v_1v_2v_3v_4$ with
$v_i \in \mathcal{H}$ over four days, $t = 1, 2, 3, 4$, can easily be computed
by multiplying the initial probability to the transition probabilities.

The above formality is too limited for many applications since the states do
not correspond to the actual observable events. To overcome this problem a
distinction between observables and states is made. The observables are a
probabilistic function of the state. Given
$\mathcal{O} = {o_1, o_2, \dots, o_M}$ being a set of $M$ observables an
emission matrix $B = {\{b_{ij}\}}_{1 \le i \le N}^{1 \le j \le M}$ with
$\sum_{j=1}^M b_{ij} = 1$ is defined, where $b_{ij}$ is the probability of
observing $o_j$ when the system is in state $h_i$. The set of states,
$\mathcal{H}$, can now be called ``hidden'' states since they are not
observable. An example of an HMM is shown in
figure~\ref{fig:hidden-markov-model}. The figure is similar to
figure~\ref{fig:markov-chain} except for the probabilities added for each
hidden state.

\begin{figure}
  \centering
  \tikzsetnextfilename{HMM}
\begin{tikzpicture} [
  ->,
  >=stealth',
  auto,
  node distance=3cm,
  main node/.style={rectangle,draw,rounded corners,align=center},
  ]

  \node[main node] (h1) {$h_1$\\{\scriptsize%
    $\begin{aligned}
        o_1 &= 0.5\\ o_2 &= 0.3\\ o_3 &= 0.2\\
    \end{aligned}$}};
  \node[main node, right of=h1] (h2) {$h_2$\\{\scriptsize%
    $\begin{aligned}
        o_1 &= 0.8\\ o_2 &= 0.1\\ o_3 &= 0.1\\
    \end{aligned}$}};
  \node[left of=h1] (p1) {};
  \node[right of=h2] (p2) {};

  \path
    (h1) edge[bend right]               node[below] {$a_{12}$} (h2)
         edge[loop below, looseness=10] node        {$a_{11}$} (h1)
    (h2) edge[bend right]               node[above] {$a_{21}$} (h1)
         edge[loop below, looseness=10] node        {$a_{22}$} (h2);

  \path
    (p1) edge node[above] {$\pi_1$} (h1)
    (p2) edge node[above] {$\pi_2$} (h2);
\end{tikzpicture}

%%% Local Variables:
%%% mode: latex
%%% TeX-master: "../master"
%%% End:

  \caption{An example of a hidden Markov model.}
  \label{fig:hidden-markov-model}
\end{figure}

For modeling weather we can now use the fact that the observables (rainy,
cloudy, and sunny weather) depend on the air pressure. This can be modeled as
$h_1$ being low pressure, $h_2$ being high pressure, $o_1$ being rainy, $o_2$
being cloudy, and $o_3$ being sunny weather. As before the probability of a
sequence of observations $Y_{1:4} = y_1y_2y_3y_4$ with $y_i \in \mathcal{O}$
can be found, but it is a harder than before since an observation sequence can be
generated in many ways.

\citet{rabiner1989tutorial} states three key problems for using hidden Markov
models in real world applications, with the first one being the one just
mentioned.
\begin{enumerate}
\item Given the observation sequence $Y_{1:T} = y_1y_2\dots{}y_T$, and the
  model $\lambda = (A, B, \Pi)$, how we compute $\Pr(O \mid \lambda)$, the
  probability of the observation sequence.
\item Given the observation sequence $Y_{1:T} = y_1y_2\dots{}y_T$, how do we
  choose a state sequence $I = v_1v_2\dots{}v_T$ which is optimal in some
  meaningful sense.
\item How we adjust the model parameters $\lambda = (A, B, \Pi)$ to maximize
  $\Pr(O \mid \lambda)$.
\end{enumerate}

The first problem of finding the joint probability of an observation sequence
$Y_{1:T}$ is solved by the Forward algorithm. This was implemented in zipHMMlib
by \citet{sand2013ziphmmlib}.

For solving the second problem (also called HMM decoding), many
methods may exist as it depends on which optimality criterion is chosen.
However, two widely used methods exist. One solution is to look at all possible
sequences of states emitting the observation sequence and pick the most likely
one. This problem is solved efficiently by the Viterbi algorithm. Another
method is to choose the states $v_t$ that are individually most likely. In this
way the expected number of correct states is maximized, but the state sequence
might not be valid, since the transition probability from $v_t$ to $v_{t + 1}$
might be zero. This is also called the posterior decoding and is solved by
Forward-Backward algorithm.

Different solutions also exist to the third problem. One solution is Baum-Welch
training that is an iterative procedure that maximizes the probability of the
observation sequence by iteratively re-estimating the parameters such that the
new model is more likely to generate the observation sequence than the previous
model.

\section{Related Work}

\citet{lifshits2009speeding} present a method for speeding up the dynamic
programming algorithms used with HMMs, namely the forward-backward algorithms
and the Viterbi algorithm. The approach is based on finding repeated substrings
in the observation input sequence. These substrings are found using five
different algorithms: Four Russians method, run length encoding, Lempel-Ziv
parsing, grammar-based compression and byte pair encoding. In the article the
forward-backward and the Viterbi algorithms are reformulated into a series of
matrix multiplications. The overall idea is that the repeated substrings
correspond to repeated matrix multiplications and by finding the repeated
substrings multiplications are avoided. A single experiment has been
performed on DNA sequences showing a speedup of the Viterbi algorithm when
using an improved Lempel-Ziv parsing without backtracking. No further
experiments have been made for the other algorithms and no code has been made
available.

\citet{sand2013ziphmmlib} presents zipHMMlib, a highly optimized HMM library for
speeding up the forward algorithm. The speed up is achieved by finding repeated
substrings using byte pair encoding. Much of the theory in this paper relies on
\cite{lifshits2009speeding}, but is extended to make the computations
numerically stable. Furthermore, the code is available as an open source
library with bindings for both Python and R.

This thesis extends the work made by \citet{sand2013ziphmmlib} to also include
a highly efficient Viterbi algorithm and a posterior decoding algorithm based
on byte pair encoding as developed theoretically by
\citet{lifshits2009speeding}. The theory will also be extended a bit to make
the computation numerically stable and the experiments will be more extensive.

%%% Local Variables:
%%% mode: latex
%%% TeX-master: "master"
%%% End:
