\chapter{Background and Related Work}
\label{cha:backgr-relat-work}

\fxfatal{Something about HMMs in general.}

\citet{lifshits2009speeding} present a method for speeding up the dynamic
programming algorithms used with HMMs, namely the forward-backward algorithms
and the Viterbi algorithm. The approach is based on finding repeated substrings
in the observed input sequence. These substrings are found using five different
algorithms: Four Russians method, run length encoding, Lempel-Ziv parsing,
grammer-based compression and byte pair encoding. The forward-backward and the
Viterbi algorithms are the reformulated into series of matrix
multiplications. The overall idea is that the repeated substrings correspond to
repeated matrix multiplications and by finding the repeated substrings the
multiplications can be avoided. Unfortunately, the code has not been made
public available and the number of experiments and the quality of these is
quite limited.

\citet{sand2013ziphmmlib} present zipHMMlib, a highly optimized HMM library for
speeding up the forward algorithm. Much of the theory in this paper rely on
\cite{lifshits2009speeding}, but it is extended to make the computations
numerically stable. Furthermore, the code is available as an open source
library with bindings for both Python and R.

This thesis extends the work made by \citet{sand2013ziphmmlib} to also include
a highly efficient Viterbi algorithm based on the theory developed by
\citet{lifshits2009speeding}, but as \citet{sand2013ziphmmlib} the theory will
also be extended a bit to make the computation numerically stable. Furthermore,
the experiments will be more extensive, thus exploiting some cases in which
the original formulation of the algorithm will be more efficient.

%%% Local Variables:
%%% mode: latex
%%% TeX-master: "master"
%%% End:
