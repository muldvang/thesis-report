\chapter{Background and Related Work}
\label{cha:backgr-relat-work}

\fxwarning{Something about HMMs in general.}

\citet{lifshits2009speeding} present a method for speeding up the dynamic
programming algorithms used with HMMs, namely the forward-backward algorithms
and the Viterbi algorithm. The approach is based on finding repeated substrings
in the observed input sequence. These substrings are found using five different
algorithms: Four Russians method, run length encoding, Lempel-Ziv parsing,
grammer-based compression and byte pair encoding. The forward-backward and the
Viterbi algorithms are the reformulated into series of matrix
multiplications. The overall idea is that the repeated substrings correspond to
repeated matrix multiplications and by finding the repeated substrings the
multiplications can be avoided. The article is primarily theoretical. A single
experiment have been performed on some DNA sequences showing a speed up of the
Viterbi algorithm when using an improved Lempel-Ziv parsing without
backtracking. No experiments have been made for the other algorithms and no
code has been made available.

\citet{sand2013ziphmmlib} present zipHMMlib, a highly optimized HMM library for
speeding up the forward algorithm. The speed up is achieved by finding repeated
substrings using byte pair encoding. Much of the theory in this paper rely on
\cite{lifshits2009speeding}, but it is extended to make the computations
numerically stable. Furthermore, the code is available as an open source
library with bindings for both Python and R.

This thesis extends the work made by \citet{sand2013ziphmmlib} to also include
a highly efficient Viterbi algorithm based on byte pair encoding as developed
theoretically by \citet{lifshits2009speeding}. The theory will also be extended
a bit to make the computation numerically stable and the experiments will be
more extensive.

%%% Local Variables:
%%% mode: latex
%%% TeX-master: "master"
%%% End:
