% chktex-file 9
% chktex-file 24
% chktex-file 44
\chapter{Experiments}
\label{cha:experiments}

This chapter covers description and discussion of all experiments performed in
this thesis. First, the used data is described followed by the experimental
setup and then the Viterbi and posterior algorithms are discussed in turn.

\section{Data}

Hidden Markov models have successfully been applied to various kinds of
data.  As this thesis is written with no specific kind of data in mind, aside
from repetitive data of course, the experiments use random data that does not
contain many repetitions and Fibonacci words that is a highly repetitive kind
of sequence.  The two types of sequences used is now described in turn.

\begin{description}
\item[Random sequences] are generated by one of the models (described below) by
  making random transitions and emissions according to the probabilities
  specified by the parameters of that model.  The generated sequences varies in
  length from $10^3$ to approximately $10^7$.
\item[Fibonacci words] are generated in a way very similar to how Fibonacci
  numbers are generated by concatenating the two previous words, corresponding
  to generating the next Fibonacci number from the two previous ones. Let $S_0$
  be ``0'' and $S_1$ be ``01''. Now define $S_n=S_{n-1}S_{n-2}$. In contrast to
  the random sequences, Fibonacci words contain many repetitions due to the
  recursive definition and compress very well using byte-pair encoding.
\end{description}

The models used are random, fully connected models, that is there is a
transition from a state to any other state. The emission probabilities are
random too. The alphabet size has been kept constant at four. In general, the
larger the alphabet, the worse compression ratio is expected. The size of the
models, i.e.\ the number of states, have been varied from 2 to 512.

\section{Experimental Setup}

The experiments have been run on a PC with an Intel Xeon W3550 $3.07$ GHz CPU
and 4 GB RAM running GNU/Linux 3.13. As zipHMM uses the ATLAS as BLAS
implementation as default, this has also been used for the experiment. ATLAS
version 3.10.2 was used.

Each point in the plots in this chapter corresponds to the mean of multiple
runs. For all experiments, the executions of a program with a set of parameters
have been repeated five times and the mean computed to even out fluctuations in
the running time. Furthermore, in the case of random sequences, 20 random
sequences of the same length has been generated and the mean of the measure
calculated, since especially for short sequences the compression ratio might vary
a lot. Furthermore, the standard deviation has been calculated and is shown as
error bars in the plots to visually indicate the fluctuations in running time.

\section{Compression Ratio}
\label{sec:compression-ratio}

The running time of the Viterbi algorithm and the indexed posterior decoding
algorithm implemented in zipHMMlib is highly dependent on the size of the
compressed sequence. As mentioned in section~\ref{sec:compr-stopp-crit} it is
possible to compress the sequence to a single character, even though this might
not be a good idea in terms of performance. To see how well the different kinds
of data compresses when used by zipHMMlib the compression ratio for all the
sequences described above have been measured. The compression ratio is defined
as $\frac{\text{original size}}{\text{compressed size}}$. Hence, the greater
compression ratio the better. Recall, that the amount of compression depends on
the estimate of how many times the algorithm is run after compressing the
sequence. For this experiment an estimates of $e = 1$ and $e = 500$ have been
provided to the program.

\begin{figure}
  \centering\ref{compression_legend}
  \begin{subfigure}[b]{0.5\textwidth}
    \centering \tikzsetnextfilename{compression_ratio_one}
\begin{tikzpicture}[trim axis left,trim axis right]
  \begin{groupplot}[
    group style={
      group name=my plots,
      group size=1 by 3,
      xlabels at=edge bottom,
      xticklabels at=edge bottom,
      vertical sep=0pt,
      every plot/.style={
        xmin=1024,
        enlargelimits=true
      }
    },
    xlabel=Sequence length,
    xmode=log,
    width=\textwidth,
    ]

    \nextgroupplot[ymode=log, height=40mm, width=\textwidth]

    \addplot[only marks, color=red, error bars, y dir=both, y explicit]
    table[x=T, y=one_compression_ratio, y error=one_compression_ratio_std]
    {plot_data/viterbi_fib_sequence_transformed.data};

    \addplot[only marks, color=blue, error bars, y dir=both, y explicit]
    table[x=T, y=one_compression_ratio, y error=one_compression_ratio_std]
    {plot_data/viterbi_1_sequence_transformed.data};

    \nextgroupplot[height=40mm, width=\textwidth, ylabel=Compression ratio]

    \addplot[only marks, color=green, error bars, y dir=both, y explicit]
    table[x=T, skip coords between index={0}{1}, y=one_compression_ratio, y error=one_compression_ratio_std]
    {plot_data/viterbi_2_sequence_transformed.data};

    \nextgroupplot[height=40mm, width=\textwidth,]

    \addplot[only marks, color=yellow, error bars, y dir=both, y explicit]
    table[x=T, y=one_compression_ratio, y error=one_compression_ratio_std]
    {plot_data/viterbi_sequence_transformed.data};

  \end{groupplot}
\end{tikzpicture}
%%% Local Variables:
%%% mode: latex
%%% TeX-master: "../master"
%%% End:

    \caption{Estimate $e = 1$.}
  \end{subfigure}%
  \begin{subfigure}[b]{0.5\textwidth}
    \centering \begin{tikzpicture}[trim axis left,trim axis right]
  \begin{axis}[
    xlabel=Sequence length,
    ylabel=Compression ratio,
    xmode=log,
    ymode=log,
    ]
    \addplot[only marks, color=colorbrewer1]
    table[x=T, y=compression_ratio]
    {plot_data/viterbi_fib_sequence_transformed.data};

    \addplot[only marks, color=colorbrewer2]
    table[x=T, y=compression_ratio]
    {plot_data/viterbi_sequence_transformed.data};

    \addplot[only marks, color=colorbrewer3]
    table[x=T, y=compression_ratio]
    {plot_data/viterbi_2_sequence_transformed.data};

    \addplot[only marks, color=colorbrewer4]
    table[x=T, y=compression_ratio]
    {plot_data/viterbi_1_sequence_transformed.data};

    \addplot[only marks, color=colorbrewer5]
    table[x=T, y=compression_ratio]
    {plot_data/viterbi_dna_sequence_transformed.data};

    \legend{Fibonacci words, Random ($\lvert{\Sigma}\rvert = 4$), Random
      ($\lvert{\Sigma}\rvert  = 2$), Unary, DNA}
  \end{axis}
\end{tikzpicture}

%%% Local Variables:
%%% mode: latex
%%% TeX-master: "../master"
%%% End:

    \caption{Estimate $e = 500$.}
  \end{subfigure}
  \caption{The compression ratio of Fibonacci words, random sequences of
    alphabet size 2, random sequences of alphabet size 4, and single character
    sequences.}
  \label{fig:compression_ratio}
\end{figure}

As seen in figure~\ref{fig:compression_ratio} the compression ratio grows
linearly for the Fibonacci words as the sequence length increases, whereas is
it only slightly growing for random sequences. The compression of Fibonacci
words are much better than the compression of random sequences, just as
expected. This suggests that the performance of the Viterbi algorithm will be
very dependent on the input sequences. Just to verify that the Fibonacci words
is very repetitive, unary sequences have also been compressed, and as seen the
compression of these two types is very comparable.

Also it is seen that the compression is very dependent of the $e$ parameter.
For Fibonacci words the compression ratio is approximately factor 10 lower when
$e = 1$ compared to $e = 500$. For random sequences only a single iteration of
the compression is run when $e = 1$ and the sequence is only slightly
compressed. For $e = 500$ more iterations of the compression is made and
thereby a better compression ratio is obtained.

For this experiment only, random sequences of an alphabet of size 2 was also
tested. It is expected that the random sequences with an alphabet of size 2
will compress two times better than the sequence with an alphabet size 4, since
the most common pair of symbols will occur twice as many times in the sequences
with alphabet size 2 than the sequences with alphabet size 4. This is also seen
in the plot for $e = 500$. For random sequences of alphabet size four the compression ratio
is approximately four, while it is almost eight for sequences with alphabet
size 2. This suggests that using the library with e.g.\ protein sequences that
has an alphabet size of 20 might not result in a good performance as good as seen
in the following experiment.

To keep the experiments simple and not introduce overly many plots, an alphabet
size of four have been used for the rest of the experiments. This corresponds
to DNA sequences that HMM's are often used with in the field of bioinformatics.

\section{Preprocessing}

As described in section~\ref{sec:saving-compr-sequ} the compression of the
input sequence is computed prior to executing the Viterbi or Forward-Backward
algorithms and can be saved to disk. This is useful if the algorithms are used with different
models, as the compression of the sequence does not depend on the model.

The preprocessing has a theoretical running time of
$O(( \lvert\mathcal{O'}\rvert - \lvert{\mathcal{O}}\rvert) T)$ as stated in
section~\ref{sec:running-time}. In figure~\ref{fig:pre_viterbi_T} the running
time divided by the length of the sequence, $T$, is shown for random sequences
and Fibonacci words of varying length.

Recall that the preprocessing iterates over the observation sequence and finds
the most common pair of symbols, introduces a new symbol, and replaces the pair
with the new symbol.

As seen the running time is not linear in $T$ for random sequences. Since the
compression ratio is dependent on $T$ as seen in the previous section and the
size of the new alphabet $\lvert\mathcal{O'}\rvert$ grows with the compression
ratio, $\lvert\mathcal{O'}\rvert$ is also dependent on $T$. Hence, when $T$
grows, $\lvert\mathcal{O'}\rvert$ also grows. Since the length of the
observation sequence only becomes a bit smaller for each iteration of the
preprocessing, while $\mathcal{O'}$ grows, what is seen in the figure is then
not surprising; the running time of the preprocessing for random sequences is
superlinear in $T$ and not linear as the theoretical running time suggests.

However, for Fibonacci words the running time is linear in $T$. For Fibonacci
words, the new alphabet size $\mathcal{O'}$ is also dependent on $T$: in the
previous section it was seen that it scales linearly. Due to the recursive
structure of Fibonacci words, the length of the observation sequence is almost
halved each time a new symbol is introduced. Hence, the first iteration of the
preprocessing takes time proportional to $T$, the next to $T/2$, the next to
$T/4$, and so on. So even though $\mathcal{O'}$ is dependent on $T$, the
recursive structure makes the running time linear in $T$ since
$T + T/2 + T/4 + T/8 + \cdots < 2T$.

\begin{figure}
  \centering
  \begin{subfigure}[b]{0.5\textwidth}
    \centering \tikzsetnextfilename{pre_viterbi_T}
\begin{tikzpicture}[trim axis left,trim axis right]
  \begin{axis}[
    xlabel=Sequence length $T$,
    ylabel=$\frac{\text{Running time}}{T}$,
    xmode=log,
    width=\textwidth
    ]
    \addplot[only marks, color=black, error bars, y dir=both, y explicit]
    table[x=T, y=many_pre_time/T, y error=many_pre_time/T_std]
    {plot_data/viterbi_sequence_transformed.data};

    \addplot[forget plot, only marks, color=black, error bars, y dir=both, y explicit]
    table[x=T, y=many_pre_time/T, y error=many_pre_time/T_std]
    {plot_data/viterbi_sequence_extended_transformed.data};
  \end{axis}
\end{tikzpicture}

%%% Local Variables:
%%% mode: latex
%%% TeX-master: "../master"
%%% End:

    \caption{Random sequences.}
  \end{subfigure}%
  \begin{subfigure}[b]{0.5\textwidth}
    \centering \tikzsetnextfilename{pre_viterbi_fib_T}
\begin{tikzpicture}[trim axis left,trim axis right]
  \begin{axis}[
    xlabel=Sequence length $T$,
    xmode=log,
    width=\textwidth,
    ]
    \addplot[only marks, color=black, error bars, y dir=both, y explicit]
    table[x=T, y=many_pre_time/T, y error=many_pre_time/T_std]
    {plot_data/viterbi_fib_sequence_transformed.data};
  \end{axis}
\end{tikzpicture}

%%% Local Variables:
%%% mode: latex
%%% TeX-master: "../master"
%%% End:

    \caption{Fibonacci words.}
  \end{subfigure}
  \caption{Preprocessing time divided by observation sequence length for
    varying lengths.}
  \label{fig:pre_viterbi_T}
\end{figure}

\section{Viterbi}

This section covers the experiments made in context of the Viterbi algorithm.
First, the theoretical asymptotic running time of the algorithm is verified to
make sure that the running time of the algorithm behaves as expected. Secondly,
a simple implementation of the classical Viterbi algorithm is compared to the
zipHMMlib implementation.

\subsection{Verification of Theoretical Running Times}
\label{sec:theor-runn-times}

In this section the theoretical running times are verified to make sure that
the implementation runs as expected. The running time has been measured for
three variations of the Viterbi algorithm corresponding to the backtracking
discussed in section~\ref{sec:backtracking}. They are named as follows.
\begin{description}
\item[Viterbi\textsubscript{L}] compresses the sequence and only computes the
  probability,
\item[Viterbi\textsubscript{P}] also backtracks,
\item[Viterbi\textsubscript{PM}] also saves memory on backtracking.
\end{description}
The next two sections verifies that the implementation follows the theoretical
running times.

\subsubsection{Sequence Length}

The implementation of the Viterbi algorithm supports both computing the Viterbi
path and not doing it. For the two of those there is a difference in the
theoretical time, since computing the path is linear in $T$. For a list of
running times for the algorithms, see table~\ref{tab:running-time}.

In figure~\ref{fig:assymptotic_viterbi_T} the running divided by the length of
the compressed sequence $T'$ is shown for input sequences of varying lengths.
The experiment has been made for both random sequences (to the left) and
Fibonacci words (to the right). As expected the fraction is close to constant
with a slightly decreasing curve for Viterbi\textsubscript{L} in both cases.
For random sequences the running time for Viterbi\textsubscript{P} and
Viterbi\textsubscript{PM} is in practice linear in $T'$ also. This is due to
the compression ratio being only approximately four, and that the backtracking
is computed quickly using the $R$ matrices. For Fibonacci words the compression
ratio is much better and the Viterbi\textsubscript{P} and
Viterbi\textsubscript{PM} algorithms are clearly not linear in $T'$.

\begin{figure}
  \centering\ref{test}
  \begin{subfigure}[b]{0.5\textwidth}
    \centering \tikzsetnextfilename{assymptotic_viterbi_T}
\begin{tikzpicture}[baseline, trim axis left,trim axis right]
  \begin{axis}[
    xlabel=Sequence length $T$,
    ylabel=$\frac{\text{Running time}}{T'}$,
    ylabel near ticks,
    every y tick scale label/.style={at={(0,1)},anchor=north east, font=\scriptsize,
      inner xsep=2pt, inner ysep=0pt},
    enlarge y limits=0.2,
    xmode=log,
    width=\textwidth,
    ]
    \addplot[viterbi-l, error bars, y dir=both, y explicit]
    table[x=T, y=many_running_time/T', y error=many_running_time/T'_std]
    {plot_data/viterbi_sequence_transformed.data};

    \addplot[viterbi-p, error bars, y dir=both, y explicit]
    table[x=T, y=many_path_running_time/T', y error=many_path_running_time/T'_std]
    {plot_data/viterbi_sequence_transformed.data};

    \addplot[viterbi-pm, error bars, y dir=both, y explicit]
    table[x=T, y=many_path_memory_running_time/T', y error=many_path_memory_running_time/T'_std]
    {plot_data/viterbi_sequence_transformed.data};
  \end{axis}
\end{tikzpicture}

%%% Local Variables:
%%% mode: latex
%%% TeX-master: "../master"
%%% End:

    \captionsetup{margin=10pt}
    \caption{For random data the running time is in practice linear in $T'$ as
      the compression ratio is approximately four.}
  \end{subfigure}%
  \begin{subfigure}[b]{0.5\textwidth}
    \centering \tikzsetnextfilename{assymptotic_viterbi_fib_T}
\begin{tikzpicture}[baseline, trim axis left,trim axis right]
  \begin{axis}[
    xlabel=Sequence length $T$,
    xmode=log,
    yticklabel style={
      /pgf/number format/fixed,
    },
    width=\textwidth,
    legend entries={Viterbi\textsubscript{L}, Viterbi\textsubscript{P}, Viterbi\textsubscript{PM}},
    legend to name=test,
    legend columns=-1,
    legend style={/tikz/every even column/.append style={column sep=10pt}},
    ]
    \addplot[viterbi-l, error bars, y dir=both, y explicit]
    table[x=T, y=many_running_time/T', y error=many_running_time/T'_std]
    {plot_data/viterbi_fib_sequence_transformed.data};

    \addplot[forget plot, viterbi-l, error bars, y dir=both, y explicit]
    table[x=T, y=many_running_time/T', y error=many_running_time/T'_std]
    {plot_data/viterbi_fib_sequence_extended_transformed.data};

    \addplot[viterbi-p, error bars, y dir=both, y explicit]
    table[x=T, y=many_path_running_time/T', y error=many_path_running_time/T'_std]
    {plot_data/viterbi_fib_sequence_transformed.data};

    \addplot[viterbi-pm, error bars, y dir=both, y explicit]
    table[x=T, y=many_path_memory_running_time/T', y error=many_path_memory_running_time/T'_std]
    {plot_data/viterbi_fib_sequence_transformed.data};
  \end{axis}
\end{tikzpicture}

%%% Local Variables:
%%% mode: latex
%%% TeX-master: "../master"
%%% End:

    \captionsetup{margin=10pt}
    \caption{For the very repetitive Fibonacci words, $T' = 2$. This exploits
      that the backtracking algorithms are far from linear in $T'$.}
  \end{subfigure}
  \caption{The running time of Viterbi algorithms divided by the compressed
    sequence length $T'$. The running time of Viterbi\textsubscript{L} is
    linear in $T'$ whereas Viterbi\textsubscript{P} and
    Viterbi\textsubscript{PM} are superlinear.}
  \label{fig:assymptotic_viterbi_T}
\end{figure}

Both backtracking methods have a theoretical running time linear in the length
of the original sequence $T$. To verify this the running time is divided by $T$
for sequences of varying lengths and shown in
figure~\ref{fig:assymptotic_viterbi_backtrack_T}. As seen, it is indeed linear
in $T$.

\begin{figure}
  \centering\ref{test2}
  \begin{subfigure}[b]{0.5\textwidth}
    \centering \tikzsetnextfilename{assymptotic_viterbi_backtrack_T}
\begin{tikzpicture}[trim axis left,trim axis right]
  \begin{axis}[
    xlabel=Sequence length $T$,
    ylabel=$\frac{\text{Running time}}{T}$,
    xmode=log,
    width=\textwidth,
    ]
    % \addplot[only marks, color=red, error bars, y dir=both, y explicit]
    % table[x=T, y=many_running_time/T, y error=many_running_time/T_std]
    % {plot_data/viterbi_sequence_transformed.data};

    \addplot[viterbi-p]
    table[x=T, y=many_path_running_time/T, y error=many_path_running_time/T_std]
    {plot_data/viterbi_sequence_transformed.data};

    \addplot[viterbi-pm, error bars, y dir=both, y explicit]
    table[x=T, y=many_path_memory_running_time/T, y error=many_path_memory_running_time/T_std]
    {plot_data/viterbi_sequence_transformed.data};
  \end{axis}
\end{tikzpicture}

%%% Local Variables:
%%% mode: latex
%%% TeX-master: "../master"
%%% End:

    \caption{Random sequences.}
  \end{subfigure}%
  \begin{subfigure}[b]{0.5\textwidth}
    \centering \tikzsetnextfilename{assymptotic_viterbi_fib_backtrack_T}
\begin{tikzpicture}[trim axis left,trim axis right]
  \begin{axis}[
    xlabel=Sequence length $T$,
    xmode=log,
    width=\textwidth,
    legend entries={% Viterbi\textsubscript{L},
      Viterbi\textsubscript{P}, Viterbi\textsubscript{PM}},
    legend to name=test2,
    legend columns=-1,
    legend style={/tikz/every even column/.append style={column sep=10pt}},
    ]
    % \addplot[only marks, color=red, error bars, y dir=both, y explicit]
    % table[x=T, y=many_running_time/T, y error=many_running_time/T_std]
    % {plot_data/viterbi_fib_sequence_transformed.data};

    \addplot[viterbi-p, error bars, y dir=both, y explicit]
    table[x=T, y=many_path_running_time/T, y error=many_path_running_time/T_std]
    {plot_data/viterbi_fib_sequence_transformed.data};

    \addplot[viterbi-pm, error bars, y dir=both, y explicit]
    table[x=T, y=many_path_memory_running_time/T, y error=many_path_memory_running_time/T_std]
    {plot_data/viterbi_fib_sequence_transformed.data};
  \end{axis}
\end{tikzpicture}

%%% Local Variables:
%%% mode: latex
%%% TeX-master: "../master"
%%% End:

    \caption{Fibonacci words.}
  \end{subfigure}
  \caption{The running time of Viterbi as a function of the original sequence
    length $T$. The running time of all three variations of the Viterbi
    algorithm are linear in $T$.}
  \label{fig:assymptotic_viterbi_backtrack_T}
\end{figure}

\subsubsection{Model Size}

The theoretical running times of all three variations of the Viterbi algorithm
are cubic in the number of states $N$. To verify this an experiment with random
sequences of length 10\,000 have been run for models with between 2 and 1024
states. Fibonacci words has not been used in this experiment since the running
time does not change with the compression. The running time divided by $N^3$ is
shown in figure~\ref{fig:assymptotic_viterbi_backtrack_N}. It is expected that
the points in the plot will form a decreasing curve. \fxerror{Discuss that
  the plot is ugly as hell. It is probably due to the alignment of the
  matrices. In an ideal world, the matrices should have two representations to
  align well in memory. This is however also expensive. Check for page faults
  when the simple programs have been implemented.}

\begin{figure}
  \centering
  \tikzsetnextfilename{assymptotic_viterbi_backtrack_N}
\begin{tikzpicture}[trim axis left,trim axis right]
  \begin{axis}[
    xlabel=Model size $N$,
    ylabel=$\frac{\text{Running time}}{N^3}$,
    xmode=log,
    % ymode=log,
    ]
    \addplot[only marks, color=colorbrewer1, error bars, y dir=both, y explicit]
    table[x=N, y=many_running_time/N^3, y error=many_running_time/N^3_std]
    {plot_data/viterbi_model_transformed.data};

    \addplot[only marks, color=colorbrewer2, error bars, y dir=both, y explicit]
    table[x=N, y=many_path_running_time/N^3, y error=many_path_running_time/N^3_std]
    {plot_data/viterbi_model_transformed.data};

    \addplot[only marks, color=colorbrewer3, error bars, y dir=both, y explicit]
    table[x=N, y=many_path_memory_running_time/N^3, y error=many_path_memory_running_time/N^3_std]
    {plot_data/viterbi_model_transformed.data};

    \legend{Viterbi\textsubscript{L}, Viterbi\textsubscript{P}, Viterbi\textsubscript{PM}}
  \end{axis}
\end{tikzpicture}

%%% Local Variables:
%%% mode: latex
%%% TeX-master: "../master"
%%% End:

  \caption{The running time of Viterbi as a function of the number of hidden
    states in the model $N$. The running time of all three variations of the Viterbi
    algorithm are cubic in $N$.}
  \label{fig:assymptotic_viterbi_backtrack_N}
\end{figure}

Recall that the running time is cubic in $N$ is due to the computation of the
$C$ matrices. The multiplication of the these matrices, i.e.\ the computation
of the Viterbi or forward-backward algorithm, takes time
proportional to $N^2$ as stated in section~\ref{sec:running-time}. When the
size of the new alphabet $M'$ is small compared to the length of the compressed
sequence, the $N^3$ term vanishes in practice when compared to the $N^2$
term. Hence, it is seen that the running time of the matrix multiplication
(and the backtracking) is proportional to $N^2$ by not compressing the
sequence. This is shown in figure~\ref{fig:assymptotic_viterbi_N} where
$\lvert M' \rvert = 4$ and $T = 10.000$.

\begin{figure}
  \centering
  \tikzsetnextfilename{assymptotic_viterbi_N}
\begin{tikzpicture}[trim axis left,trim axis right]
  \begin{axis}[
    xlabel=Model size $N$,
    ylabel=$\frac{\text{Running time}}{N^2}$,
    xmode=log,
    % ymode=log,
    ]
    \addplot[only marks, color=colorbrewer1, error bars, y dir=both, y explicit]
    table[x=N, y=uncompressed_running_time/N^2, y error=uncompressed_running_time/N^2_std]
    {plot_data/viterbi_model_transformed.data};

    \addplot[only marks, color=colorbrewer2, error bars, y dir=both, y explicit]
    table[x=N, y=uncompressed_path_running_time/N^2, y error=uncompressed_path_running_time/N^2_std]
    {plot_data/viterbi_model_transformed.data};

    \addplot[only marks, color=colorbrewer3, error bars, y dir=both, y explicit]
    table[x=N, y=uncompressed_path_memory_running_time/N^2, y error=uncompressed_path_memory_running_time/N^2_std]
    {plot_data/viterbi_model_transformed.data};

    \legend{Viterbi\textsubscript{L}, Viterbi\textsubscript{P}, Viterbi\textsubscript{PM}}
  \end{axis}
\end{tikzpicture}

%%% Local Variables:
%%% mode: latex
%%% TeX-master: "../master"
%%% End:

  \caption{The running time of Viterbi as a function of the number of hidden
    states in the model $N$. If the input sequences are not compressed, the
    running time of all three variations of the Viterbi algorithm are quadratic
    in $N$.}
  \label{fig:assymptotic_viterbi_N}
\end{figure}

This ends the experiments concerning the theoretical running times of the
Viterbi algorithm itself. It has been shown that the running time in practice
in terms of the length of the observation sequence is very dependent on the
structure of it, but it still fits the theoretical worst case running time. In
terms of the model size the physical hardware becomes visible. However, when
the models get large enough the running time stabilizes and the theoretical
running time is confirmed. \fxerror{Is this correct?}

\subsection{Comparing zipHMMlib Viterbi to the Classical Viterbi Algorithm}
\label{sec:comp-ziphmml-viterbi}

The zipHMMlib Viterbi algorithm is compared to an implementation of the
classical Viterbi algorithm. Recall that the compressed sequence from the
preprocessing can be saved to disk for later use. For some problems though one
might only need to run Viterbi once for a sequence. As described in
section~\ref{sec:compr-stopp-crit} the preprocessing uses an estimate $e$ of
how many times the Viterbi algorithm is run to compress the sequence
appropriately.  To experiment with this, experiments have been made with
$e = 1$ and $e = 500$ executions of Viterbi.

The algorithms are compared to their respective counterpart of the classical
algorithm, e.g.\ for Viterbi\textsubscript{L} algorithms the classical
algorithm without backtracking has been used. In the following plots a
``speedup factor'' is shown on the y-axis. The speedup factor is calculated as
the running time of the classical algorithm (with or without without backtracking
enabled) divided by the running time of algorithm being measured, including the
time it takes to do the preprocessing. Hence, any speedup factor larger than one
means that the algorithm being measured is faster than the classical algorithm.

\subsubsection{Random Data}

The first experiments have been made using random sequences.

In figures~\ref{fig:compressed_1_speedup_vs_sequence_length}
and~\ref{fig:compressed_500_speedup_vs_sequence_length} the speedup factor for
the Viterbi algorithm is shown for sequences of varying length with $e = 1$ and
$e = 500$, respectively. A model with 16 states was used for these experiments.
As expected, computing the Viterbi path limits the speedup factor since the
running time is linear in $T$ instead of $T'$ as it is for only computing the
probability. The Viterbi\textsubscript{P} and Viterbi\textsubscript{PM} gain a
similar speedup. This is expected as the main part of the memory saving in
Viterbi\textsubscript{PM} essentially consists in computing the Viterbi twice.
Since this is computed for both the zipHMM implementation and the classical
implementation, the speedup becomes the same as for the
Viterbi\textsubscript{P}. For $e = 500$ the speedup is better than for $e = 1$
due to the compression ratio. For small sequences however, the speedup factor
is equal for $e = 1$ and $e = 500$. This is due to the zipHMM Viterbi algorithm
computing the $C_{o_i}$ matrices before starting the actual Viterbi
computation. As the sequences become longer this computation becomes a smaller
fraction of the running time and thereby an increasing running time is seen for
$e = 500$. For $e = 1$ the new alphabet size $\mathcal{O'}$ is very small and
the $C_{o_i}$ matrices are quickly computed. Hence, the running time is almost
independent on this computation and the speedup becomes close to constant for
increasing sequence lengths.

\begin{figure}
  \fxwarning{Why is the running time decreasing a bit?}
  \centering
  \tikzsetnextfilename{compressed_1_speedup_vs_sequence_length}
\begin{tikzpicture}[trim axis left,trim axis right]
  \begin{axis}[
    xlabel=Sequence length $T$,
    ylabel=Speedup factor,
    ymax=10,
    enlarge y limits=true,
    xmode=log,
    ]
    \addplot[viterbi-l, error bars, y dir=both, y explicit]
    table[x=T, y=one_total_ratio, y error=one_total_ratio_std]
    {plot_data/viterbi_sequence_transformed.data};

    \addplot[forget plot, viterbi-l, error bars, y dir=both, y explicit]
    table[x=T, y=one_total_ratio, y error=one_total_ratio_std]
    {plot_data/viterbi_sequence_extended_transformed.data};

    \addplot[viterbi-p, error bars, y dir=both, y explicit]
    table[x=T, y=one_path_total_ratio, y error=one_path_total_ratio_std]
    {plot_data/viterbi_sequence_transformed.data};

    \addplot[viterbi-pm, error bars, y dir=both, y explicit]
    table[x=T, y=one_path_memory_total_ratio, y error=one_path_memory_total_ratio_std]
    {plot_data/viterbi_sequence_transformed.data};

    \legend{Viterbi\textsubscript{L}, Viterbi\textsubscript{P}, Viterbi\textsubscript{PM}}
  \end{axis}
\end{tikzpicture}

%%% Local Variables:
%%% mode: latex
%%% TeX-master: "../master"
%%% End:

  \caption{The speed up factor for the Viterbi algorithms with $e = 1$ of the
    total running time for sequences of varying lengths using a HMM with 16
    states.}
  \label{fig:compressed_1_speedup_vs_sequence_length}
\end{figure}

\begin{figure}
  \centering
  \tikzsetnextfilename{compressed_500_speedup_vs_sequence_length}
\begin{tikzpicture}[trim axis left,trim axis right]
  \begin{axis}[
    xlabel=Sequence length $T$,
    ylabel=Speedup factor,
    xmode=log,
    ]
    \addplot[viterbi-l, error bars, y dir=both, y explicit]
    table[x=T, y=many_total_ratio, y error=many_total_ratio_std]
    {plot_data/viterbi_sequence_transformed.data};

    \addplot[viterbi-p, error bars, y dir=both, y explicit]
    table[x=T, y=many_path_total_ratio, y error=many_path_total_ratio_std]
    {plot_data/viterbi_sequence_transformed.data};

    \addplot[viterbi-pm, error bars, y dir=both, y explicit]
    table[x=T, y=many_path_memory_total_ratio, y error=many_path_memory_total_ratio_std]
    {plot_data/viterbi_sequence_transformed.data};

    \legend{Viterbi\textsubscript{L}, Viterbi\textsubscript{P}, Viterbi\textsubscript{PM}}
  \end{axis}
\end{tikzpicture}

%%% Local Variables:
%%% mode: latex
%%% TeX-master: "../master"
%%% End:

  \caption{The speed up factor for the Viterbi algorithms with $e = 500$ of the
    total running time for sequences of varying lengths using a HMM with 16
    states.}
  \label{fig:compressed_500_speedup_vs_sequence_length}
\end{figure}

Instead of presenting the speedup factor as a function of the sequence length,
the model size is used in the next experiment. So while the size of the model
is varied the length of the observation sequence is kept constant at length
10\,000. Recall that the running time of Viterbi is $O(M' N^3 + N^2 T')$. The
cubic term might become a problem as $N$ grows. The speedup factor as function
of model size $N$ is plotted in figures~\ref{fig:speedup_vs_N}
and~\ref{fig:speedup_vs_N2} for $e = 1$ and $e = 500$,
respectively.

\begin{figure}
  \centering
  \tikzsetnextfilename{speedup_vs_N}
\begin{tikzpicture}[trim axis left,trim axis right]
  \begin{axis}[
    ylabel near ticks,
    width=\textwidth,
    xlabel=Number of states $N$,
    ylabel=Speedup factor,
    xmode=log,
    ymax=2.5,
    enlarge y limits=true,
    legend entries={Viterbi\textsubscript{L}, Viterbi\textsubscript{P}, Viterbi\textsubscript{PM}},
    legend to name=legend:speedup_vs_N,
    ]
    \addplot[viterbi-l, error bars, y dir=both, y explicit]
    table[x=N, y=one_total_ratio, y error=one_total_ratio_std]
    {plot_data/viterbi_model_transformed.data};

    \addplot[viterbi-p, error bars, y dir=both, y explicit]
    table[x=N, y=one_path_total_ratio, y error=one_path_total_ratio_std]
    {plot_data/viterbi_model_transformed.data};

    \addplot[viterbi-pm, error bars, y dir=both, y explicit]
    table[x=N, y=one_path_memory_total_ratio, y error=one_path_memory_total_ratio_std]
    {plot_data/viterbi_model_transformed.data};
  \end{axis}
\end{tikzpicture}

%%% Local Variables:
%%% mode: latex
%%% TeX-master: "../master"
%%% End:

  \caption{The speedup factor for varying model sizes using the Viterbi
    algorithms and $e = 1$.}
  \label{fig:speedup_vs_N}
\end{figure}

\begin{figure}
  \centering
  \tikzsetnextfilename{speedup_vs_N2}
\begin{tikzpicture}[trim axis left,trim axis right]
  \begin{axis}[
    ylabel near ticks,
    width=\textwidth,
    xlabel=Number of states $N$,
    xmode=log,
    ymax=12,
    enlarge y limits=true,
    ]
    \addplot[viterbi-l, error bars, y dir=both, y explicit]
    table[x=N, y=many_total_ratio, y error=many_total_ratio_std]
    {plot_data/viterbi_model_transformed.data};

    \addplot[viterbi-p, error bars, y dir=both, y explicit]
    table[x=N, y=many_path_total_ratio, y error=many_path_total_ratio_std]
    {plot_data/viterbi_model_transformed.data};

    \addplot[viterbi-pm, error bars, y dir=both, y explicit]
    table[x=N, y=many_path_memory_total_ratio, y error=many_path_memory_total_ratio_std]
    {plot_data/viterbi_model_transformed.data};
  \end{axis}
\end{tikzpicture}

%%% Local Variables:
%%% mode: latex
%%% TeX-master: "../master"
%%% End:

  \caption{The speedup factor for varying model sizes using the Viterbi
    algorithms and $e=500$.}
  \label{fig:speedup_vs_N2}
\end{figure}

For $e = 1$ the speedup factor is growing in the beginning as the number of
states increases. This is due to the matrix representation of the algorithm
making it more efficient. However, for larger matrices, the cubed $N$ term
becomes a limiting factor and the speedup factor drops below 1. This is also
seen for $e = 500$. For $e = 500$ however, the best result is obtained for very
small models. This is due to the new alphabet size $\mathcal{O'}$ in being
bigger when $e = 500$. Hence, the $M' N^3$ term of the running time becomes
more dominating and the speedup becomes smaller even for small increases of the
model size.

\subsubsection{Fibonacci Words}

Speedups were gained for many cases using random data, but it is
expected that much greater speedups will be obtained by using repetitive
sequences. Thus, experiments similar to the ones for random data has been
conducted. In figures~\ref{fig:fib_compressed_1_speedup_vs_sequence_length}
and~\ref{fig:fib_compressed_500_speedup_vs_sequence_length} the speedup factor
is shown for sequences of varying length using $e = 1$ and $e = 500$, respectively.

\begin{figure}
  \centering
  \tikzsetnextfilename{fib_compressed_1_speedup_vs_sequence_length}
\begin{tikzpicture}[trim axis left,trim axis right]
  \begin{axis}[
    ylabel near ticks,
    width=\textwidth,
    xlabel=Sequence length $T$,
    ylabel=Speedup factor,
    xmode=log,
    legend entries={Viterbi\textsubscript{L}, Viterbi\textsubscript{P}, Viterbi\textsubscript{PM}},
    legend to name=legend:fib_compressed_speedup_vs_sequence_length,
    ]
    \addplot[viterbi-l, error bars, y dir=both, y explicit]
    table[x=T, y=one_total_ratio, y error=one_total_ratio_std]
    {plot_data/viterbi_fib_sequence_transformed.data};

    \addplot[forget plot, viterbi-l, error bars, y dir=both, y explicit]
    table[x=T, y=one_total_ratio, y error=one_total_ratio_std]
    {plot_data/viterbi_fib_sequence_extended_transformed.data};

    \addplot[viterbi-p, error bars, y dir=both, y explicit]
    table[x=T, y=one_path_total_ratio, y error=one_path_total_ratio_std]
    {plot_data/viterbi_fib_sequence_transformed.data};

    \addplot[viterbi-pm, error bars, y dir=both, y explicit]
    table[x=T, y=one_path_memory_total_ratio, y error=one_path_memory_total_ratio_std]
    {plot_data/viterbi_fib_sequence_transformed.data};
  \end{axis}
\end{tikzpicture}

%%% Local Variables:
%%% mode: latex
%%% TeX-master: "../master"
%%% End:

  \caption{The speed up factor for the Viterbi algorithms with $e = 1$ of the total
    running time for Fibonacci words of varying
    lengths using a HMM with 16 states.}
  \label{fig:fib_compressed_1_speedup_vs_sequence_length}
\end{figure}

\begin{figure}
  \centering
  \tikzsetnextfilename{fib_compressed_500_speedup_vs_sequence_length}
\begin{tikzpicture}[trim axis left,trim axis right]
  \begin{axis}[
    ylabel near ticks,
    width=\textwidth,
    xlabel=Sequence length $T$,
    xmode=log,
    ymode=log,
    ]
    \addplot[viterbi-l, error bars, y dir=both, y explicit]
    table[x=T, y=many_total_ratio, y error=many_total_ratio_std]
    {plot_data/viterbi_fib_sequence_transformed.data};

    \addplot[forget plot, viterbi-l, error bars, y dir=both, y explicit]
    table[x=T, y=many_total_ratio, y error=many_total_ratio_std]
    {plot_data/viterbi_fib_sequence_extended_transformed.data};

    \addplot[viterbi-p, error bars, y dir=both, y explicit]
    table[x=T, y=many_path_total_ratio, y error=many_path_total_ratio_std]
    {plot_data/viterbi_fib_sequence_transformed.data};

    \addplot[viterbi-pm, error bars, y dir=both, y explicit]
    table[x=T, y=many_path_memory_total_ratio, y error=many_path_memory_total_ratio_std]
    {plot_data/viterbi_fib_sequence_transformed.data};
  \end{axis}
\end{tikzpicture}

%%% Local Variables:
%%% mode: latex
%%% TeX-master: "../master"
%%% End:

  \caption{The speed up factor for the Viterbi algorithms with $e = 500$ of the total
    running time for Fibonacci words of varying
    lengths using a HMM with 16 states.}
  \label{fig:fib_compressed_500_speedup_vs_sequence_length}
\end{figure}

As seen, the speedup factors are greater than for random data. It is however
very dependent on the value of $e$. For $e = 1$, even though the sequence is
being compressed a lot as seen in section~\ref{sec:compression-ratio}, the
speedup factor is limited since a lot of time is still spent on the
compression. For $e = 500$ however, the speedup factor stays increasing for the
Viterbi\textsubscript{L} algorithm. For random data the computation of the
Viterbi path limited the speedup factor. This limitation becomes larger for
Fibonacci words. This is seen in
figure~\ref{fig:fib_compressed_500_speedup_vs_sequence_length}, where the
computation of the Viterbi path makes the computation orders of magnitudes
slower than just computing the probability. However, the speedup factor for the
Viterbi\textsubscript{P} and Viterbi\textsubscript{PM} algorithms still perform
better on Fibonacci words than on random sequences.

\subsubsection{Future Work}

To get an impression of where the bottleneck of the algorithm is, the running
time of the preprocessing and the execution of the Viterbi algorithm has been
measured. To keep the number of graphs in this chapter at a reasonable level,
the experiment has only been made for the Viterbi\textsubscript{PM} algorithms.
Similar results can be obtained using the Viterbi\textsubscript{L} and
Viterbi\textsubscript{P} algorithms. Again, for $e = 1$ the preprocessing has
been computed and then Viterbi has been executed once. For $e = 500$ the
preprocessing has been computed followed by 500 executions of Viterbi. The
result is seen in figure~\ref{fig:pre_vs_running}.

\begin{figure}
  \centering\ref{named}
  \begin{subfigure}{0.5\textwidth}
    \centering \tikzsetnextfilename{viterbi_pre_vs_running_one}
\begin{tikzpicture}[trim axis left,trim axis right]
  \begin{axis}[
    stack plots=y,
    area style,
    ylabel near ticks,
    every y tick scale label/.style={at={(0,1)},anchor=north east, font=\scriptsize,
      inner xsep=2pt, inner ysep=0pt},
    enlarge y limits=false,
    enlarge x limits=false,
    xmode=log,
    ymin=0,
    ymax=1,
    width=\textwidth,
    ]

    \addplot
    table[x=T, y=one_path_memory_pre_fraction]
    {plot_data/viterbi_sequence_transformed.data}
    \closedcycle;

    \addplot
    table[x=T, y=one_path_memory_running_fraction]
    {plot_data/viterbi_sequence_transformed.data}
    \closedcycle;
  \end{axis}
\end{tikzpicture}

%%% Local Variables:
%%% mode: latex
%%% TeX-master: "../master"
%%% End:

    \caption{Random data, $e = 1$.}
  \end{subfigure}%
  \begin{subfigure}{0.5\textwidth}
    \centering \tikzsetnextfilename{viterbi_pre_vs_running_many}
\begin{tikzpicture}[trim axis left,trim axis right]
  \begin{axis}[
    legend entries={Preprocessing, Viterbi},
    legend to name=named,
    legend columns=-1,
    legend style={/tikz/every even column/.append style={column sep=10pt}},
    stack plots=y,
    area style,
    enlarge x limits=false,
    enlarge y limits=false,
    xmode=log,
    ymin=0,
    ymax=1,
    width=\textwidth,
    ]

    \addplot[fill=my-blue, draw=my-blue, fill opacity=0.3]
    table[x=T, y=many_path_memory_pre_fraction]
    {plot_data/viterbi_sequence_transformed.data}
    \closedcycle;

    \addplot[fill=my-red, draw=my-red, fill opacity=0.3]
    table[x=T, y=many_path_memory_running_fraction]
    {plot_data/viterbi_sequence_transformed.data}
    \closedcycle;
  \end{axis}
\end{tikzpicture}

%%% Local Variables:
%%% mode: latex
%%% TeX-master: "../master"
%%% End:

    \caption{Random data, $ e = 500$.}
  \end{subfigure}

  \begin{subfigure}{0.5\textwidth}
    \centering \tikzsetnextfilename{viterbi_pre_vs_running_one_path_memory}
\begin{tikzpicture}[trim axis left,trim axis right]
  \begin{axis}[
    legend style={at={(0, 1)},anchor=north west},
    stack plots=y,
    area style,
    enlarge x limits=false,
    xmode=log,
    ymin=0,
    ymax=1,
    width=\textwidth,
    ]

    \addplot
    table[x=T, y=one_path_memory_pre_fraction]
    {plot_data/viterbi_sequence_transformed.data}
    \closedcycle;

    \addplot
    table[x=T, y=one_path_memory_running_fraction]
    {plot_data/viterbi_sequence_transformed.data}
    \closedcycle;

    % \legend{Preprocessing, Viterbi};
  \end{axis}
\end{tikzpicture}

%%% Local Variables:
%%% mode: latex
%%% TeX-master: "../master"
%%% End:

    \caption{Fibonacci words, $e = 1$.}
  \end{subfigure}%
  \begin{subfigure}{0.5\textwidth}
    \centering \tikzsetnextfilename{viterbi_pre_vs_running_many_path}
\begin{tikzpicture}[trim axis left,trim axis right]
  \begin{axis}[
    legend style={at={(0, 1)},anchor=north west},
    stack plots=y,
    area style,
    enlarge x limits=false,
    enlarge y limits=false,
    xmode=log,
    ymin=0,
    ymax=1,
    width=\textwidth,
    ]

    \addplot[fill=my-blue, draw=my-blue, fill opacity=0.3]
    table[x=T, y=many_path_memory_pre_fraction]
    {plot_data/viterbi_fib_sequence_transformed.data} \closedcycle;

    \addplot[fill=my-red, draw=my-red, fill opacity=0.3]
    table[x=T, y=many_path_memory_running_fraction]
    {plot_data/viterbi_fib_sequence_transformed.data} \closedcycle;

    % \legend{Preprocessing, Viterbi};
  \end{axis}
\end{tikzpicture}

%%% Local Variables:
%%% mode: latex
%%% TeX-master: "../master"
%%% End:

    \caption{Fibonacci words, $ e = 500$.}
  \end{subfigure}
  \caption{Fraction of the running time spent on preprocessing and the
    Viterbi\textsubscript{PM} algorithm for input sequences of varying length.}
  \label{fig:pre_vs_running}
\end{figure}

For the $e = 1$ experiments it is seen that a large part of the running time is
spent on the preprocessing. However, as the number of executions is increased
the time of the preprocessing becomes lower for random data. For $e = 500$ the
time spent on preprocessing is very low compared to the actual computation of
the Viterbi algorithm. In terms of performance gains, making the preprocessing
more efficient will only give a speedup gain when the number of executions of
the Viterbi algorithm is low and primarily for smaller sequences, while higher
efficiency in the Viterbi algorithm or in the effectiveness of the compression
will make the running time smaller for many executions.

\subsubsection{Conclusion}

This ends the section of experiments for the Viterbi algorithm. In
conclusion the compression of sequences does speed up the execution time of the
Viterbi algorithm in many cases. The best results are of course found for
repetitive data with the Viterbi algorithm run multiple times and the Viterbi
path is not computed. Speedup factors in the order of hundreds were found, so
for these kinds of problems this method has a great potential. If the Viterbi
path is requested the speedup factors becomes much lower. The maximum speedup
seen in the experiments were below 20. In other scenarios with random data a
more modest speedup was obtained. For random data and request of the Viterbi
path speedup factors below 1 is obtained. Hence, the zipHMM Viterbi is slower
than the classical Viterbi algorithm in some cases.

\section{Posterior Decoding}

This section contains all experiments performed on the posterior decoding
algorithm. As described in section~\ref{sec:probl-expl-repet}, no efficient way
of exploiting sequence repetitions has been found. Nevertheless, experiments
have been made to see whether a minor speedup can be obtained by computing the
posterior decoding in the matrix multiplications framework in zipHMM.\@ In
contrast to the previous section with multiple variations of the Viterbi
algorithm, only a single variation makes sense here: the input sequence is not
compressed and the path is computed.

The section is split into two parts. The first part compares the actual
running time to the theoretical running time to verify that the implementation
of the algorithm satisfies the theory. The second part compares it to the
classical implementation of posterior decoding.

\subsection{Asymptotic Running Time}
\label{sec:asympt-runn-time}

Like in section~\ref{sec:theor-runn-times}, the theoretical compared the actual
running times experiments have been made using random data. The theoretical
running time is $O(M N^3 + TN^2)$. Since $M$ is very small compared to $T$, is
expected that the first term vanishes. This was already seen for the Viterbi
algorithm in section~\ref{sec:theor-runn-times},
figure~\ref{fig:assymptotic_viterbi_N}. Hence, for this experiment the
theoretical running time $O(TN^2)$ is assumed. This of course only makes sense
if $N$ is not too large, but as seen in the plots the assumption of a
theoretical running time of $O(TN^2)$ does make sense for these experiments.

First, the running time compared to the sequence length is shown in
figure~\ref{fig:posterior_T}. A model with 16 states has been used. As expected,
the running time is decreasing going towards constant.

\begin{figure}
  \centering
  \tikzsetnextfilename{assymptotic_posterior_T}
\begin{tikzpicture}[trim axis left,trim axis right]
  \begin{axis}[
    xlabel=Sequence length $T$,
    ylabel=$\frac{\text{Running time}}{T}$,
    xmode=log,
    ]
    \addplot[only marks, color=black, error bars, y dir=both, y explicit]
    table[x=T, y=zipHMMlib_time/T, y error=zipHMMlib_time/T_std]
    {plot_data/posterior_sequence_transformed.data};
  \end{axis}
\end{tikzpicture}

%%% Local Variables:
%%% mode: latex
%%% TeX-master: "../master"
%%% End:

  \caption{The running time of posterior decoding is linear in the size of the
    input sequence.}
  \label{fig:posterior_T}
\end{figure}

Secondly, in figure~\ref{fig:posterior_N} the running time for an increasing
number of states is shown. A random sequence of length $10.000$ has been used. It
is seen that the running time is indeed quadratic in the numbers of states
in practice.

\begin{figure}
  \centering
  \tikzsetnextfilename{assymptotic_posterior_N}
\begin{tikzpicture}[trim axis left,trim axis right]
  \begin{axis}[
    xlabel=Model size $N$,
    ylabel=$\frac{\text{Running time}}{N^2}$,
    xmode=log,
    ]
    \addplot[only marks, color=black, error bars, y dir=both, y explicit]
    table[x=N, y=zipHMMlib_time/N^2, y error=zipHMMlib_time/N^2_std]
    {plot_data/posterior_model_transformed.data};
  \end{axis}
\end{tikzpicture}

%%% Local Variables:
%%% mode: latex
%%% TeX-master: "../master"
%%% End:

  \caption{The running time of posterior decoding is quadratic in the number of
    states.}
  \label{fig:posterior_N}
\end{figure}

It is concluded that the actual running time of the algorithm follows the
theoretical.

\subsection{Comparison to the Classical Algorithm}
\label{sec:comp-class-algor}

Now the algorithm is compared the classical algorithm. In
figure~\ref{fig:posterior_speedup_vs_sequence_length} the speedup factor as
function of the sequence length is shown. Random sequences and a model with 16
states was used for the experiment. In all tested cases the algorithm is a bit
faster than the classical implementation. There is a tendency that the speedup
factor becomes larger as the input sequence grows. This is due to the initial
setup of the algorithm that slows down the algorithm a bit, which was also seen
in figure~\ref{fig:posterior_T}, where the factor between running time and
observation sequence length is larger for small sequences than for large
sequence. Hence, as the sequence length is increasing the initial setup time
becomes relatively smaller and the speedup factor is increasing. Speedup
factors between 1 and 5 are gained in this experiment. This might change if the
number of states in the model is different.

\begin{figure}
  \centering
  \begin{tikzpicture}[trim axis left,trim axis right]
  \begin{axis}[
    xlabel=Sequence length,
    ylabel=Speedup factor,
    xmode=log,
    ]
    \addplot[only marks, color=colorbrewer2, error bars, y dir=both, y explicit]
    table[x=n, y=zipHMMlib_ratio, y error=zipHMMlib_ratio_std]
    {plot_data/posterior_sequence_transformed.data};

    \legend{Uncompressed Path}
  \end{axis}
\end{tikzpicture}

%%% Local Variables:
%%% mode: latex
%%% TeX-master: "../master"
%%% End:

  \caption{The speed up factor for sequences of varying lengths using a HMM
    with 16 states.}
  \label{fig:posterior_speedup_vs_sequence_length}
\end{figure}

Looking at the speedup factor as function of the number of states in
figure~\ref{fig:posterior_speedup_vs_N}, it is seen that for small models it is
below 1. This means that the algorithm is slower than the classical posterior
decoding algorithm. However, as the number of states in the model increases the
speedup factor quickly becomes larger than 1, due to the matrix multiplications
being very efficient due to the BLAS framework.

\begin{figure}
  \centering
  \tikzsetnextfilename{posterior_speedup_vs_N}
\begin{tikzpicture}[trim axis left,trim axis right]
  \begin{axis}[
    xlabel=Model size $N$,
    ylabel=Speedup factor,
    xmode=log,
    max space between ticks=20pt,
    ]
    \addplot[only marks, color=colorbrewer2, error bars, y dir=both, y explicit]
    table[x=N, y=zipHMMlib_ratio, y error=zipHMMlib_ratio_std]
    {plot_data/posterior_model_transformed.data};

    \legend{Uncompressed Path}
  \end{axis}
\end{tikzpicture}

%%% Local Variables:
%%% mode: latex
%%% TeX-master: "../master"
%%% End:

  \caption{The speed up factor as function of the model size using a sequence
    of length 10\,000.}
  \label{fig:posterior_speedup_vs_N}
\end{figure}

These experiments end the section of the posterior decoding. Overall the
algorithm has a running time that is very comparable to the classical
implementation of posterior decoding. For scenarios with large models and
sequences the zipHMM is faster whereas the classical algorithm is a better
for small models.

\section{Indexed Posterior Decoding}

In this section experiments on the indexed posterior decoding is
made. Recall that indexed posterior decoding also takes two indices $i$ and
$j$ as input and only returns the posterior decoding for the subsequence
$Y_{i:j}$.

Recall from section~\ref{sec:running-time-2} that the algorithm has a
theoretical running time of $O(M' N^3 + N^2 T' + N^2 (j - i + \log T))$,
assuming the input sequence has already been compressed. This is very similar
to both the Viterbi algorithm and the posterior decoding algorithm. Therefore,
the verification of the theoretical running in practice has been omitted, since
most of the implementation uses the same code as the two previously tested
algorithms. However, the three terms of the running time are still discussed in
turn below.

The first term $M' N^3$ comes from the computation of the $C$ matrices. This
has already been experimented with for the Viterbi algorithm in
section~\ref{sec:theor-runn-times}. Since the only difference between the
computations of the matrices for the Viterbi algorithm and the forward-backward
algorithm is matrix multiplication vs.\ max times multiplication and numerical
stability, the theoretical running time will also hold for the indexed posterior
decoding.

The second term $N^2 T'$ comes from computing the forward and backward tables
for the compressed sequence of length $T'$. In
section~\ref{sec:asympt-runn-time} this was already verified the for posterior
decoding algorithm.

Finally, the $N^2 (j - i + \log T)$ term comes from the decompression of the
compressed subsequence that is linear in $j - i + \log T$ and from computing
the posterior decoding for the subsequence. The running time of the posterior
decoding has already been shown to match the theoretical running time. The
decompression has a quite simple implementation and experiments have not been
made for this. \fxwarning{Make experiment for decompression? It would also be
  interesting in terms of comparing random data to Fibonacci words.}

\subsection{Comparison to the Classical Algorithm}

The indexed posterior decoding algorithm has been compared to the classical
posterior decoding algorithm. For the classical algorithm the entire posterior
decoding has been found and then the state sequence from index $i$ to $j$ has
been extracted.

As for the experimentation of the Viterbi algorithm, the comparison to the
classical implementation is first made for random data and then for Fibonacci
words. Like for the Viterbi algorithm the algorithm is analyzed by varying the
input sequence length and the model size, but the length of the partially
decompressed subsequence $Z$ also has an impact on the running time, so this
has also been experimented with.

\subsection{Random Data}

For the first experiment the distance between the two indices $i$ and $j$ has
been varied for random input sequences of length 10\,000 and a model with 16
states. For this experiment the ``extra'' symbols corresponding to the indices
$[k, i)$ and $(j, l]$ can be left out as there is at most $2 \times \log 10\,000$
of these according to the analysis in section~\ref{sec:running-time-2}, which
is a small number compared to $j - i$.

Depending on how well the data compresses, the value of the length of the
compressed sequence $T'$ will be larger or smaller than the length of
subsequence $Z$. For random data that only has a compression ratio of
approximately 4 as seen section~\ref{sec:compression-ratio} the number of
``extra'' symbols will be small. Hence, when $j - i$ is small the length of $Z$
will also be small compared to $T'$. Therefore, it is expected that $T'$ will
dominate the running time and small increases of $j - i$ will not make major
changes to the overall running time. However, for large values of $j - i$, the
value will be close to or larger than $T'$ and the value will have a larger
impact on the running time, thus making the running time higher.

This is seen in
figure~\ref{fig:assymptotic_indexed_posterior_subseq_length.tex} where the
running time is nearly constant for small values of $j - i$, but starts to grow
when $j - i$ gets closer to $T$, since the forward and backward tables then are
computed for a large part of the uncompressed observation sequence.

\begin{figure}
  \centering
  \tikzsetnextfilename{assymptotic_indexed_posterior_subseq_length}
\begin{tikzpicture}[trim axis left,trim axis right]
  \begin{axis}[
    xlabel=$j - i$,
    ylabel=Running time (per iteration),
    xmode=log,
    ]
    \addplot[indexed-one, error bars, y dir=both, y explicit]
    table[x=subseq_length, y=one_total_time, y error=one_total_time_std]
    {plot_data/indexed_posterior_subseq_length_transformed.data};

    \addplot[indexed-many, error bars, y dir=both, y explicit]
    table[x=subseq_length, y=many_total_time, y error=many_total_time_std]
    {plot_data/indexed_posterior_subseq_length_transformed.data};

    \legend{Indexed posterior $e = 1$, Indexed posterior $e = 500$}
  \end{axis}
\end{tikzpicture}

%%% Local Variables:
%%% mode: latex
%%% TeX-master: "../master"
%%% End:

  \caption{The running time of indexed posterior decoding for varying distances
    between $i$ and $j$. Random sequences has been used.}
  \label{fig:assymptotic_indexed_posterior_subseq_length.tex}
\end{figure}

In figure~\ref{fig:indexed_posterior_speedup_vs_subseq} the running time is
compared the classical algorithm. A speedup is gained when running the
algorithm multiple times, but for a single run the algorithm is slower than the
classical one in this case. The speedup get smaller when $j - i$ gets closer to
$T$. That is expected from the running time seen in
figure~\ref{fig:assymptotic_indexed_posterior_subseq_length.tex} that was
discussed in the previous paragraph.

\begin{figure}
  \centering
  \tikzsetnextfilename{indexed_posterior_speedup_vs_subseq}
\begin{tikzpicture}[trim axis left,trim axis right]
  \begin{axis}[
    xlabel=$j - i$,
    ylabel=Speedup factor,
    xmode=log,
    ]
    \addplot[indexed-one, error bars, y dir=both, y explicit]
    table[x=subseq_length, y=one_total_ratio, y error=one_total_ratio_std]
    {plot_data/indexed_posterior_subseq_length_transformed.data};

    \addplot[indexed-many, error bars, y dir=both, y explicit]
    table[x=subseq_length, y=many_total_ratio, y error=many_total_ratio_std]
    {plot_data/indexed_posterior_subseq_length_transformed.data};

    \legend{Indexed posterior $e = 1$, Indexed posterior $e = 500$}
  \end{axis}
\end{tikzpicture}

%%% Local Variables:
%%% mode: latex
%%% TeX-master: "../master"
%%% End:

  \caption{The indexed posterior decoding algorithm compared to the simple
    algorithm for random data for varying distances between $i$ and $j$.}
  \label{fig:indexed_posterior_speedup_vs_subseq}
\end{figure}

For the second experiment the length of the input sequence has been varied from
small sequences of length $10^3$ to approximately $10^7$. As already seen in
sections~\ref{sec:comp-ziphmml-viterbi} and~\ref{sec:comp-class-algor} the
longer the input sequence, the better speedup for the Viterbi algorithm and the
posterior decoding. It is expected that the indexed posterior
decoding will be similar to the posterior decoding algorithm. The
result of this experiment is shown in
figure~\ref{fig:indexed_posterior_speedup_vs_T}. The result is as
expected. Again, the indexed posterior decoding 1 algorithm is slower than
the classical algorithm for small sequences. For the indexed posterior
decoding 500 algorithm however, speedups of up to factor 15 is seen in this
example.

\begin{figure}
  \centering
  \tikzsetnextfilename{indexed_posterior_speedup_vs_T}
\begin{tikzpicture}[trim axis left,trim axis right]
  \begin{axis}[
    xlabel=Sequence length $T$,
    ylabel=Speedup factor,
    xmode=log,
    ]
    \addplot[only marks, color=colorbrewer1, error bars, y dir=both, y explicit]
    table[x=T, y=one_total_ratio, y error=one_total_ratio_std]
    {plot_data/indexed_posterior_sequence_transformed.data};

    \addplot[only marks, color=colorbrewer2, error bars, y dir=both, y explicit]
    table[x=T, y=many_total_ratio, y error=many_total_ratio_std]
    {plot_data/indexed_posterior_sequence_transformed.data};

    \legend{Indexed posterior 1, Indexed posterior 500}
  \end{axis}
\end{tikzpicture}

%%% Local Variables:
%%% mode: latex
%%% TeX-master: "../master"
%%% End:

  \caption{The indexed posterior decoding algorithm compared to the simple
    algorithm for random data for varying input sequence lengths.}
  \label{fig:indexed_posterior_speedup_vs_T}
\end{figure}

Finally, the number of states in the model has been varied. Sequences of length
10\,000 were used and the posterior decoding has been computed for an indexed of
length 200. As for the Viterbi algorithm experiments in
section~\ref{sec:comp-ziphmml-viterbi} it is expected that the algorithm is
most efficient for ``medium sized'' models. As seen in
figure~\ref{fig:indexed_posterior_speedup_vs_N} that is also the case. The
result is very similar to the results seen in figures~\ref{fig:speedup_vs_N}
and~\ref{fig:speedup_vs_N2}.

\begin{figure}
  \centering
  \tikzsetnextfilename{indexed_posterior_speedup_vs_N}
\begin{tikzpicture}[trim axis left,trim axis right]
  \begin{axis}[
    xlabel=Model size $N$,
    ylabel=Speedup factor,
    xmode=log,
    ]
    \addplot[only marks, color=colorbrewer1, error bars, y dir=both, y explicit]
    table[x=N, y=one_total_ratio, y error=one_total_ratio_std]
    {plot_data/indexed_posterior_model_transformed.data};

    \addplot[only marks, color=colorbrewer2, error bars, y dir=both, y explicit]
    table[x=N, y=many_total_ratio, y error=many_total_ratio_std]
    {plot_data/indexed_posterior_model_transformed.data};

    \legend{Substring posterior 1, Substring posterior 500}
  \end{axis}
\end{tikzpicture}

%%% Local Variables:
%%% mode: latex
%%% TeX-master: "../master"
%%% End:

  \caption{The indexed posterior decoding algorithm compared to the simple
    algorithm for random data for varying model sizes.}
  \label{fig:indexed_posterior_speedup_vs_N}
\end{figure}

\subsection{Fibonacci Words}

The results of the experiments using random data showed speedups in some cases
whereas in other cases slowdowns were obtained. As seen for the Viterbi
algorithm much better results were obtained by using Fibonacci words so this
is also expected for the indexed posterior decoding.

The experiments from the previous section is repeated with Fibonacci words
instead of random sequences. It is expected that this will provide better
speedups. The result of these experiments is shown in
figures~\ref{fig:assymptotic_indexed_posterior_fib_subseq_length.tex},~\ref{fig:fib_indexed_posterior_speedup_vs_subseq},
and~\ref{fig:indexed_posterior_fib_speedup_vs_T}. As seen the curves in the
plots are similar to those of random data except that the speedup factor is
much better as expected.

\begin{figure}
  \centering
  \tikzsetnextfilename{assymptotic_indexed_posterior_fib_subseq_length}
\begin{tikzpicture}[trim axis left,trim axis right]
  \begin{axis}[
    xlabel=$j - i$,
    ylabel=Running time (per iteration),
    xmode=log,
    ]
    \addplot[indexed-one, error bars, y dir=both, y explicit]
    table[x=subseq_length, y=one_total_time, y error=one_total_time_std]
    {plot_data/indexed_posterior_fib_subseq_length_transformed.data};

    \addplot[indexed-many, error bars, y dir=both, y explicit]
    table[x=subseq_length, y=many_total_time, y error=many_total_time_std]
    {plot_data/indexed_posterior_fib_subseq_length_transformed.data};

    \legend{Indexed posterior $e = 1$, Indexed posterior $e = 500$}
  \end{axis}
\end{tikzpicture}

%%% Local Variables:
%%% mode: latex
%%% TeX-master: "../master"
%%% End:

  \caption{The running time for indexed posterior decoding for
    varying indexed lengths. Fibonacci word has been used}
  \label{fig:assymptotic_indexed_posterior_fib_subseq_length.tex}
\end{figure}

\begin{figure}
  \centering
  \tikzsetnextfilename{fib_indexed_posterior_speedup_vs_subseq}
\begin{tikzpicture}[trim axis left,trim axis right]
  \begin{axis}[
    xlabel=$j - i$,
    ylabel=Speedup factor,
    xmode=log,
    ]
    \addplot[only marks, color=colorbrewer1, error bars, y dir=both, y explicit]
    table[x=subseq_length, y=one_total_ratio, y error=one_total_ratio_std]
    {plot_data/indexed_posterior_fib_subseq_length_transformed.data};

    \addplot[only marks, color=colorbrewer2, error bars, y dir=both, y explicit]
    table[x=subseq_length, y=many_total_ratio, y error=many_total_ratio_std]
    {plot_data/indexed_posterior_fib_subseq_length_transformed.data};

    \legend{Substring posterior 1, Substring posterior 500}
  \end{axis}
\end{tikzpicture}

%%% Local Variables:
%%% mode: latex
%%% TeX-master: "../master"
%%% End:

  \caption{The indexed posterior decoding algorithm compared to the simple
    algorithm for a Fibonacci word for varying substring lengths.}
  \label{fig:fib_indexed_posterior_speedup_vs_subseq}
\end{figure}

\begin{figure}
  \centering
  \tikzsetnextfilename{indexed_posterior_fib_speedup_vs_T}
\begin{tikzpicture}[trim axis left,trim axis right]
  \begin{axis}[
    xlabel=Sequence length $T$,
    ylabel=Speedup factor,
    xmode=log,
    ymode=log,
    ]
    \addplot[indexed-one, error bars, y dir=both, y explicit]
    table[x=T, y=one_total_ratio, y error=one_total_ratio_std]
    {plot_data/indexed_posterior_fib_sequence_transformed.data};

    \addplot[indexed-many, error bars, y dir=both, y explicit]
    table[x=T, y=many_total_ratio, y error=many_total_ratio_std]
    {plot_data/indexed_posterior_fib_sequence_transformed.data};

    \legend{Indexed posterior $e = 1$, Indexed posterior $e = 500$}
  \end{axis}
\end{tikzpicture}

%%% Local Variables:
%%% mode: latex
%%% TeX-master: "../master"
%%% End:

  \caption{The indexed posterior decoding algorithm compared to the simple
    algorithm for Fibonacci words of varying lengths.}
  \label{fig:indexed_posterior_fib_speedup_vs_T}
\end{figure}

%%% Local Variables:
%%% mode: latex
%%% TeX-command-extra-options: "-shell-escape"
%%% TeX-master: "master"
%%% End:
