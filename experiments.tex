% chktex-file 24
\chapter{Experiments}
\label{cha:experiments}

\section{Viterbi}
\label{sec:viterbi}

\subsection{Assymptotic Running Times}
\label{sec:assymp-runn-times}

\subsubsection{Sequence Length $n$}
\label{sec:sequence-length-n}

\begin{figure}[H]
  \centering
  \tikzsetnextfilename{pre_viterbi_n}
\begin{tikzpicture}[trim axis left,trim axis right]
  \begin{axis}[
    xlabel=Sequence length $T$,
    ylabel=$\frac{\text{Running time}}{T}$,
    xmode=log,
    ]
    \addplot[only marks, color=black, error bars, y dir=both, y explicit]
    table[x=T, y=many_pre_time/T, y error=many_pre_time/T_std]
    {plot_data/viterbi_sequence_transformed.data};
  \end{axis}
\end{tikzpicture}

%%% Local Variables:
%%% mode: latex
%%% TeX-master: "../master"
%%% End:

  \caption{zipHMMlib preprocessing time for varying sequence lengths.}
  \label{fig:pre_viterbi_n}
\end{figure}

\begin{figure}[H]
  \centering
  \begin{tikzpicture}[trim axis left,trim axis right]
  \begin{axis}[
    xlabel=Sequence length $n$,
    ylabel=$\frac{\text{Running time}}{n}$,
    xmode=log,
    ]
    \addplot[only marks, color=colorbrewer1]
    table[x=n, y=zipHMMlib_running_time/n]
    {plot_data/sequence_transformed.data};
  \end{axis}
\end{tikzpicture}

%%% Local Variables:
%%% mode: latex
%%% TeX-master: "../master"
%%% End:

  \caption{zipHMMlib running time for varying sequence lengths.}
  \label{fig:assymptotic_viterbi_n}
\end{figure}

\begin{figure}[H]
  \centering
  \tikzsetnextfilename{assymptotic_viterbi_path_n}
\begin{tikzpicture}[trim axis left,trim axis right]
  \begin{axis}[
    xlabel=Sequence length $T$,
    ylabel=$\frac{\text{Running time}}{T}$,
    xmode=log,
    ]
    \addplot[only marks, color=colorbrewer1, error bars, y dir=both, y explicit]
    table[x=T, y=many_path_running_time/T, y error=many_path_running_time/T_std]
    {plot_data/viterbi_sequence_transformed.data};

    \addplot[only marks, color=colorbrewer2, error bars, y dir=both, y explicit]
    table[x=T, y=many_path_memory_running_time/T, y error=many_path_memory_running_time/T_std]
    {plot_data/viterbi_sequence_transformed.data};

    \legend{Normal, Space saving};
  \end{axis}
\end{tikzpicture}

%%% Local Variables:
%%% mode: latex
%%% TeX-master: "../master"
%%% End:

  \caption{zipHMMlib Path running time for varying sequence lengths.}
  \label{fig:assymptotic_viterbi_path_n}
\end{figure}

\begin{figure}[H]
  \centering
  \begin{tikzpicture}[trim axis left,trim axis right]
  \begin{axis}[
    xlabel=Sequence length $n$,
    ylabel=$\frac{\text{Running time}}{n}$,
    xmode=log,
    ymin=0,
    ]
    \addplot[only marks, color=colorbrewer1, error bars, y dir=both, y explicit]
    table[x=n, y=zipHMMlib_path_backtrack_time/n, y error=zipHMMlib_path_backtrack_time/n_std]
    {plot_data/sequence_transformed.data};
  \end{axis}
\end{tikzpicture}

%%% Local Variables:
%%% mode: latex
%%% TeX-master: "../master"
%%% End:

  \caption{zipHMMlib Path backtracking time for varying sequence lengths.}
  \label{fig:assymptotic_viterbi_backtrack_n}
\end{figure}

\subsubsection{Model Size $k$}
\label{sec:model-size-k}

\begin{figure}[H]
  \centering
  \input{figures/pre_viterbi_k}
  \caption{zipHMMlib preprocessing time for varying model sizes.}
  \label{fig:pre_viterbi_k}
\end{figure}

\begin{figure}[H]
  \centering
  \tikzsetnextfilename{assymptotic_viterbi_k}
\begin{tikzpicture}[trim axis left,trim axis right]
  \begin{axis}[
    xlabel=Model size $N$,
    ylabel=$\frac{\text{Running time}}{N^3}$,
    xmode=log,
    ]
    \addplot[only marks, color=black, error bars, y dir=both, y explicit]
    table[x=N, y=zipHMMlib_running_time/N^3, y error=zipHMMlib_running_time/N^3_std]
    {plot_data/viterbi_model_transformed.data};
  \end{axis}
\end{tikzpicture}

%%% Local Variables:
%%% mode: latex
%%% TeX-master: "../master"
%%% End:

  \caption{zipHMMlib running time for varying model sizes.}
  \label{fig:assymptotic_viterbi_k}
\end{figure}

\begin{figure}[H]
  \centering
  \begin{tikzpicture}[trim axis left,trim axis right]
  \begin{axis}[
    xlabel=Model size $k$,
    ylabel=$\frac{\text{Running time}}{k^2}$,
    xmode=log,
    ]
    \addplot[only marks, color=colorbrewer1]
    table[x=k, y=zipHMMlib_path_running_time/k^2]
    {plot_data/model_transformed.data};
  \end{axis}
\end{tikzpicture}

%%% Local Variables:
%%% mode: latex
%%% TeX-master: "../master"
%%% End:

  \caption{zipHMMlib Path running time for varying model sizes.}
  \label{fig:assymptotic_viterbi_path_k}
\end{figure}

% \begin{figure}[H]
%   \centering
%   \begin{tikzpicture}[trim axis left,trim axis right]
  \begin{axis}[
    xlabel=Model size $k$,
    ylabel=Running time,
    xmode=log,
    ]
    \addplot[only marks, color=colorbrewer1]
    table[x=k, y=zipHMMlib_path_backtrack_time]
    {plot_data/model_transformed.data};
  \end{axis}
\end{tikzpicture}

%%% Local Variables:
%%% mode: latex
%%% TeX-master: "../master"
%%% End:

%   \caption{zipHMMlib Path backtracking time for varying model sizes.}
%   \label{fig:assymptotic_viterbi_backtrack_k}
% \end{figure}

\subsection{Fibonacci Words}
\label{sec:fibonacci-words}

Using Fibonacci words for experiments are kind of a worst case example of
sequences.

\begin{figure}[H]
  \centering \fxerror{Make experiment with DNA and amino acid sequences.}
  \begin{tikzpicture}[trim axis left,trim axis right]
  \begin{axis}[
    xlabel=Sequence length,
    ylabel=Compression ratio,
    xmode=log,
    ymode=log,
    ]
    \addplot[only marks, color=colorbrewer1]
    table[x=n, y=compression_ratio]
    {plot_data/fib_sequence_transformed.data};

    \addplot[only marks, color=colorbrewer2]
    table[x=n, y=compression_ratio]
    {plot_data/sequence_transformed.data};

    \legend{Fibonacci words, Random alphabet size 4}
  \end{axis}
\end{tikzpicture}

%%% Local Variables:
%%% mode: latex
%%% TeX-master: "../master"
%%% End:

  \caption{The compression ratio of Fibonacci words, random sequences of
    alphabet size 2, random sequences of alphabet size 4, DNA
    sequences, and amino acid sequences.}
  \label{fig:compression_ratio}
\end{figure}

\subsection{Comparing zipHMMlib to the original Viterbi algorithm.}
\label{sec:comp-ziphmml-orig}

\begin{figure}[H]
  \centering
  \tikzsetnextfilename{speedup_vs_complexity}
\begin{tikzpicture}[trim axis left,trim axis right]
  \begin{axis}[
    xlabel=Frequency of 1s,
    ylabel=Speedup factor,
    xmode=log,
    ymode=log
    ]
    \addplot[only marks, color=colorbrewer1]
    table[x=frequency, y=zipHMMlib_total_ratio]
    {plot_data/complexity_transformed.data};

    \addplot[only marks, color=colorbrewer2]
    table[x=frequency, y=zipHMMlib_path_total_ratio]
    {plot_data/complexity_transformed.data};

    \addplot[only marks, color=colorbrewer3]
    table[x=frequency, y=zipHMMlib_running_ratio]
    {plot_data/complexity_transformed.data};

    \addplot[only marks, color=colorbrewer4]
    table[x=frequency, y=zipHMMlib_path_running_ratio]
    {plot_data/complexity_transformed.data};

    \legend{zipHMMlib w/ preprocessing, zipHMMlib Path w/ preprocessing, zipHMMlib wo/ preprocessing, zipHMMlib Path wo/ preprocessing}
  \end{axis}
\end{tikzpicture}

%%% Local Variables:
%%% mode: latex
%%% TeX-master: "../master"
%%% End:

  \caption{Running time vs.\ sequence complexity using sequences of length $10^6$.}
  \label{fig:speedup_vs_complexity}
\end{figure}

\begin{figure}[H]
  \centering
  \begin{tikzpicture}[trim axis left,trim axis right]
  \begin{axis}[
    xlabel=Number of states,
    ylabel=Speedup factor,
    xmode=log,
    ]
    \addplot[only marks, color=colorbrewer1, error bars, y dir=both, y explicit]
    table[x=k, y=zipHMMlib_total_ratio, y error=zipHMMlib_total_ratio_std]
    {plot_data/model_transformed.data};

    \addplot[only marks, color=colorbrewer2, error bars, y dir=both, y explicit]
    table[x=k, y=zipHMMlib_path_total_ratio, y error=zipHMMlib_path_total_ratio_std]
    {plot_data/model_transformed.data};

    % \addplot[only marks, color=colorbrewer3, error bars, y dir=both, y explicit]
    % table[x=k, y=zipHMMlib_running_ratio, y error=zipHMMlib_running_ratio_std]
    % {plot_data/model_transformed.data};

    % \addplot[only marks, color=colorbrewer4, error bars, y dir=both, y explicit]
    % table[x=k, y=zipHMMlib_path_running_ratio, y error=zipHMMlib_path_running_ratio_std]
    % {plot_data/model_transformed.data};

    \legend{zipHMMlib, zipHMMlib Path}
  \end{axis}
\end{tikzpicture}

%%% Local Variables:
%%% mode: latex
%%% TeX-master: "../master"
%%% End:

  \caption{Running time vs.\ model size.}
  \label{fig:speedup_vs_k}
\end{figure}

\begin{figure}[H]
  \centering
  \begin{tikzpicture}[trim axis left,trim axis right]
  \begin{axis}[
    xlabel=Sequence length,
    ylabel=Speedup factor,
    xmode=log,
    ]
    \addplot[only marks, color=colorbrewer1, error bars, y dir=both, y explicit]
    table[x=n, y=zipHMMlib_total_ratio, y error=zipHMMlib_total_ratio_std]
    {plot_data/sequence_transformed.data};

    \addplot[only marks, color=colorbrewer2, error bars, y dir=both, y explicit]
    table[x=n, y=zipHMMlib_path_total_ratio, y error=zipHMMlib_path_total_ratio_std]
    {plot_data/sequence_transformed.data};

    \addplot[only marks, color=colorbrewer3, error bars, y dir=both, y explicit]
    table[x=n, y=zipHMMlib_running_ratio, y error=zipHMMlib_running_ratio_std]
    {plot_data/sequence_transformed.data};

    \addplot[only marks, color=colorbrewer4, error bars, y dir=both, y explicit]
    table[x=n, y=zipHMMlib_path_running_ratio, y error=zipHMMlib_path_running_ratio_std]
    {plot_data/sequence_transformed.data};

    \legend{zipHMMlib w/ preprocessing, zipHMMlib Path w/ preprocessing, zipHMMlib wo/ preprocessing, zipHMMlib Path wo/ preprocessing}
  \end{axis}
\end{tikzpicture}

%%% Local Variables:
%%% mode: latex
%%% TeX-master: "../master"
%%% End:

  \caption{The speed up factor of the total running time include preprocessing
    for sequences of varying lengths using a HMM with 16 states.}
  \label{fig:speedup_vs_sequence_length}
\end{figure}

\begin{figure}[H]
  \centering
  \begin{tikzpicture}[trim axis left,trim axis right]
  \begin{axis}[
    xlabel=Sequence length $T$,
    ylabel=Speedup factor,
    xmode=log,
    ]
    \addplot[only marks, color=colorbrewer1, error bars, y dir=both, y explicit]
    table[x=n, y=zipHMMlib_uncompressed_running_ratio, y error=zipHMMlib_uncompressed_running_ratio_std]
    {plot_data/sequence_transformed.data};

    \addplot[only marks, color=colorbrewer2, error bars, y dir=both, y explicit]
    table[x=n, y=zipHMMlib_uncompressed_path_running_ratio, y error=zipHMMlib_uncompressed_path_running_ratio_std]
    {plot_data/sequence_transformed.data};

    \addplot[only marks, color=colorbrewer3, error bars, y dir=both, y explicit]
    table[x=n, y=zipHMMlib_running_ratio, y error=zipHMMlib_running_ratio_std]
    {plot_data/sequence_transformed.data};

    \addplot[only marks, color=colorbrewer4, error bars, y dir=both, y explicit]
    table[x=n, y=zipHMMlib_path_running_ratio, y error=zipHMMlib_path_running_ratio_std]
    {plot_data/sequence_transformed.data};

    \legend{zipHMMlib Uncompressed, zipHMMlib Uncompressed Path, zipHMMlib, zipHMMlib Path}
  \end{axis}
\end{tikzpicture}

%%% Local Variables:
%%% mode: latex
%%% TeX-master: "../master"
%%% End:

  \caption{The speed up factor of the running time excluding preprocessing for
    sequences of varying lengths using a HMM with 16 states.}
  \label{fig:speedup_vs_sequence_length2}
\end{figure}

\begin{figure}[H]
  \centering
  \tikzsetnextfilename{fib_speedup_vs_sequence_length}
\begin{tikzpicture}[trim axis left,trim axis right]
  \begin{axis}[
    xlabel=Sequence length $T$,
    ylabel=Speedup factor,
    xmode=log,
    ]
    \addplot[only marks, color=colorbrewer1, error bars, y dir=both, y explicit]
    table[x=T, y=uncompressed_total_ratio, y error=uncompressed_total_ratio_std]
    {plot_data/viterbi_fib_sequence_transformed.data};

    \addplot[only marks, color=colorbrewer2, error bars, y dir=both, y explicit]
    table[x=T, y=uncompressed_path_total_ratio, y error=uncompressed_path_total_ratio_std]
    {plot_data/viterbi_fib_sequence_transformed.data};

    \addplot[only marks, color=colorbrewer6, error bars, y dir=both, y explicit]
    table[x=T, y=uncompressed_path_memory_total_ratio, y error=uncompressed_path_memory_total_ratio_std]
    {plot_data/viterbi_fib_sequence_transformed.data};

    \addplot[only marks, color=colorbrewer3, error bars, y dir=both, y explicit]
    table[x=T, y=many_total_ratio, y error=many_total_ratio_std]
    {plot_data/viterbi_fib_sequence_transformed.data};

    \addplot[only marks, color=colorbrewer4, error bars, y dir=both, y explicit]
    table[x=T, y=many_path_total_ratio, y error=many_path_total_ratio_std]
    {plot_data/viterbi_fib_sequence_transformed.data};

    \addplot[only marks, color=colorbrewer5, error bars, y dir=both, y explicit]
    table[x=T, y=many_path_memory_total_ratio, y error=many_path_memory_total_ratio_std]
    {plot_data/viterbi_fib_sequence_transformed.data};

    \legend{Uncompressed, Uncompressed Path, Uncompressed Path Space Saving,
      Compressed, Compressed Path, Compressed Path Space Saving}
  \end{axis}
\end{tikzpicture}

%%% Local Variables:
%%% mode: latex
%%% TeX-master: "../master"
%%% End:

  \caption{The speed up factor of the total running time include preprocessing
    for Fibonacci words of varying lengths using a HMM with 16 states.}
  \label{fig:fib_speedup_vs_sequence_length}
\end{figure}

\begin{figure}[H]
  \centering
  \input{figures/fib_speedup_vs_sequence_length2.tex}
  \caption{The speed up factor of the running time excluding preprocessing for
    Fibonacci words of varying lengths using a HMM with 16 states.}
  \label{fig:fib_speedup_vs_sequence_length2}
\end{figure}

\section{Posterior Decoding}
\label{sec:posterior-decoding}

\subsection{Assymptotic Running Time}
\label{sec:assympt-runn-time}

\begin{figure}[H]
  \centering
  \input{figures/assymptotic_posterior_n.tex}
  \caption{The running time of posterior decoding is linear in the size of the
    input sequence.}
  \label{fig:posterior_n}
\end{figure}

\begin{figure}[H]
  \centering
  \tikzsetnextfilename{assymptotic_posterior_k}
\begin{tikzpicture}[trim axis left,trim axis right]
  \begin{axis}[
    xlabel=Model size $k$,
    ylabel=$\frac{\text{Running time}}{N^2}$,
    xmode=log,
    ]
    \addplot[only marks, color=black, error bars, y dir=both, y explicit]
    table[x=k, y=zipHMMlib_time/k^2, y error=zipHMMlib_time/k^2_std]
    {plot_data/posterior_model_transformed.data};
  \end{axis}
\end{tikzpicture}

%%% Local Variables:
%%% mode: latex
%%% TeX-master: "../master"
%%% End:

  \caption{The running time of posterior decoding is quadratic in the number of
    states.}
  \label{fig:posterior_k}
\end{figure}

\subsection{Comparison to the Naive Algorithm}
\label{sec:comp-naive-algor}

\begin{figure}[H]
  \centering
  \tikzsetnextfilename{posterior_speedup_vs_k}
\begin{tikzpicture}[trim axis left,trim axis right]
  \begin{axis}[
    xlabel=Number of states,
    ylabel=Speedup factor,
    xmode=log,
    ]
    \addplot[only marks, color=colorbrewer2, error bars, y dir=both, y explicit]
    table[x=k, y=zipHMMlib_ratio, y error=zipHMMlib_ratio_std]
    {plot_data/posterior_model_transformed.data};

    \legend{Uncompressed Path}
  \end{axis}
\end{tikzpicture}

%%% Local Variables:
%%% mode: latex
%%% TeX-master: "../master"
%%% End:

  \caption{The speed up factor as function of the model size using a sequence
    of length 100000.}
  \label{fig:posterior_speedup_vs_k}
\end{figure}

\begin{figure}[H]
  \centering
  \tikzsetnextfilename{posterior_speedup_vs_sequence_length}
\begin{tikzpicture}[trim axis left,trim axis right]
  \begin{axis}[
    xlabel=Sequence length,
    ylabel=Speedup factor,
    xmode=log,
    ]
    \addplot[only marks, color=colorbrewer2, error bars, y dir=both, y explicit]
    table[x=T, y=zipHMMlib_ratio, y error=zipHMMlib_ratio_std]
    {plot_data/posterior_sequence_transformed.data};

    \legend{Uncompressed Path}
  \end{axis}
\end{tikzpicture}

%%% Local Variables:
%%% mode: latex
%%% TeX-master: "../master"
%%% End:

  \caption{The speed up factor for sequences of varying lengths using a HMM
    with 16 states.}
  \label{fig:posterior_speedup_vs_sequence_length}
\end{figure}

%%% Local Variables:
%%% mode: latex
%%% TeX-master: "master"
%%% End:
