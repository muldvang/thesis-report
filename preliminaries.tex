\chapter{Preliminaries}
\label{cha:preliminaries}

In the section the definition of and notation for a Hidden Markov Model and the
Viterbi algorithm will be presented using the notation also used by
\citet{sand2013ziphmmlib}.

\section{Hidden Markov Models}
\label{sec:hidden-markov-models}

A hidden Markov model is a statistical model in which it is assumed that an
observed sequence is generated by a Markov process with unobserved hidden
states. Hence, for a sequence $Y_{1:T} = y_1y_2\dots{}y_T \in \mathcal{O*}$
generated by the model there exist one or more hidden sequences
$X_{1:T} = x_1x_2\dots{}x_T \in \mathcal{H*}$, with $\mathcal{O}$ and
$\mathcal{H}$ being finite alphabets over the observables and hidden
states. The hidden sequence may be seen as a explanation of the observed
sequence.

Formally a HMM can be defined as
\begin{itemize}
\item $\mathcal{H} = {h_1, h_2, \dots, h_N}$, a finite alphabet of hidden
  states;
\item $\mathcal{O} = {o_1, o_2, \dots, o_M}$, a finite alphabet of observables;
\item a vector $\Pi = {(\pi_i)}_{1 \le i \le N}$, where $\pi_i = \Pr(x_1 =
  h_i)$ is the probablity of the model starting in hidden state $h_i$;
\item a matrix $A = {\{a_{ij}\}}_{1 \le i \le N}$, where $a_{ij} = \Pr(x_t
  = h_j \mid x_{t - 1} = h_i)$ is the probability of a transition from state
  $h_i$ to state $h_j$;
\item a matrix $B = {\{b_{ij}\}}_{1 \le i \le N}^{1 \le j \le M}$, where
  $b_{ij} = \Pr(y_t = o_j \mid x_t = h_i)$ is the probability of state
  $h_i$ emitting $o_j$.
\end{itemize}

An HMM is parameterised by $\pi$, $A$, and $B$, which is denoted by $\lambda =
(\pi, A, B)$.

\section{The Classical Viterbi Algorithm}
\label{sec:class-viterbi-algor}

The Viterbi algorithm finds the probability of the most likely sequence of
hidden states given a model $\lambda$ and an observed sequence $Y_{1:T}$ by
maximizing the probability of the observed and hidden sequences for all
possible hidden sequences: $\Pr(Y_{1:T} \mid \lambda) = \max_{x_{1:T}}
\Pr(Y_{1:T}, X_{1:T} = x_{1:T} \mid \lambda)$. This may be computed efficiently
by filling out a table, $\omega$, with entries $\omega_t(x_t) = \Pr(Y_{1:T},
X_t = x_t \mid \lambda) = \max_{x_{1:t-1}} \Pr(Y_{1:t}, X_{1:t} = x_{1:t} \mid
\lambda)$ column by column from left to right, using the recursion
\begin{align*}
  \omega_1(x_1) &= \pi_{x_1} b_{x_1, y_1} \\
  \omega_t(x_t) &= b_{x_t, y_t} \max_{x_{t - 1}} \omega_{t - 1}(x_{t - 1})
                  a_{x_{t - 1}, x_t}.
\end{align*}
After filling out $\omega$, $\Pr(Y_{1:T} \mid \lambda)$ can be computed as
$\Pr(Y_{1:T} \mid \lambda) = \max_{x_T} \omega_T(x_T)$.

To obtain the sequence of hidden states $\omega$ is backtracked using the
recursion
\begin{align*}
  X_T &= \argmax_{x_T} \omega_T(x_T) \\
  X_{t} &= \argmax_{x_{t}} b_{X_{t + 1}, y_{t + 1}} \omega_{t}(x_t) a_{x_t, X_{t + 1}}.
\end{align*}

The space consumption of this algorithm is the size of $\omega$ which is $O(N
T)$. The time required to fill out a cell, the algorithm maximizes over all
cells in the previous column yielding a running time of $O(N^2 T)$.

%%% Local Variables:
%%% mode: latex
%%% TeX-master: "master"
%%% End:
