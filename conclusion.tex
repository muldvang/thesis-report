\chapter{Conclusion}
\label{cha:conclusion}

\citet{sand2013ziphmmlib} implements a library zipHMM---a library for working
with hidden Markov models. It includes an efficient implementation of the forward
algorithm that obtains a speedup compared to the classical implementation by
exploiting repetitions in the input sequence.

In this master's thesis the zipHMM library has been extended to include
implementations of the Viterbi, the backward and two variants of the posterior
decoding algorithms. This includes description of the theory behind the
algorithm, efficient implementations of these, and experiments that evaluates
the performance compared to the standard implementation.

By exploiting repetitions in the observation sequence the running time of the
Viterbi algorithm has changed from $O(N^2 T)$, with $N$ being the number of
states and $T$ being the length of the observation sequence, to $O(N^2 T')$ for
computing the likelihood of the most likely path and $O(N^2 T' + T)$ for also
getting the path. This requires that the observation sequence has been
preprocessed. This takes time $O(M' N^3)$ with $M'$ being the new alphabet size
that is dependent on the data and a user provided estimate of how many times
the Viterbi algorithm is executed afterwards.

The experiments show that a possibly very large speedup may be gained. However,
it is very dependent on how well the data compress, i.e.\ how many repetitions
the observation sequence contains, and how many times the user estimates that the
algorithm is executed after the compression. Furthermore, since the
preprocessing is cubic in the number of states, a speedup is not obtained in
the models used become too large.

The backward algorithm has been implemented as part of the implementation for
computing the posterior decoding. It has not been posible to find a method for
exploiting repetitions in the observation sequence, so in practice it becomes a
battle between constants whether the classical version or the zipHMMlib version
is the fastest. In the experimental setup used in this thesis the constants are
in favor of the zipHMMlib implementation, when the observation sequence length and
model size are large enough.

Exploting repetitions is however posible for posterior decoding if only a
part of the posterior decoding is requested. In that case the running time
becomes \fxfatal{Insert running time} with $j - i$ being the length of the
requested substring.

\section{Future Work}

\begin{itemize}
\item Make bindings for R and Python.
\item Implement compression stopping criterion based on whether backtracking is
  enabled or not.
\item Implement multiple substring requests without having to recompute the
  tables of the compressed sequence.
\item Implement substring Viterbi.
\end{itemize}

%%% Local Variables:
%%% mode: latex
%%% TeX-master: "master.tex"
%%% End:
