% chktex-file 44

\chapter{Conclusion}
\label{cha:conclusion}

\citet{sand2013ziphmmlib} implements a zipHMMlib for working with hidden Markov
models. It includes an efficient implementation of the forward algorithm that
obtains a speedup compared to an implementation of the original algorithm by
exploiting repetitions in the input sequence.

In this master's thesis the zipHMMlib library has been extended to include
implementations of the Viterbi, the backward and two variants of the posterior
decoding algorithms. This includes description of the theory behind the
algorithm, efficient implementations of these, and experiments that evaluates
the performance compared to the standard implementation.

By exploiting repetitions in the observation sequence the running time of the
Viterbi algorithm for computing the probability of the most likely path has
changed from $O(N^2 T)$, with $N$ being the number of states and $T$ being the
length of the observation sequence, to $O(M' N^3 + N^2 T')$ where $T'$ is the
length of the compressed observation sequence and $M'$ is the new alphabet
size. If the Viterbi path is required the running time has changed to
$O(M' N^3 + N^2 T' + T)$. This requires that the observation sequence has been
preprocessed. This takes $O((M' - M) T)$ time with $M'$ being the new alphabet size
that is dependent on the data and a user provided estimate of how many times
the Viterbi algorithm is executed afterwards.

The experiments for the Viterbi algorithm show that a possibly very large speedup can be gained. However,
it is very dependent on how well the data compresses, i.e.\ how many repetitions
the observation sequence contains, and how many times the user estimates that the
algorithm is executed after the compression. Furthermore, since the
algorithms are cubic in the number of states a speedup is not obtained for
large models and large alphabet sizes. The experiments show that a speedup
between 0.5 and 20 can be a typical result depending on how the algorithm is used.

The backward algorithm has been implemented as part of the implementation for
computing the posterior decoding. It has not been possible to find a method for
exploiting repetitions in the observation sequence, so in practice it becomes a
battle between constants whether the original version or the zipHMMlib version
of the posterior decoding algorithm is the fastest. The running time of the
original algorithm is $O(N^2 T)$ while the zipHMMlib version has a running time
of $O(MN^3 + N^2 T)$. In the experimental setup used in this thesis the
constants are in favor of the zipHMMlib implementation, when the observation
sequence length and model size are large enough. In general, a speedup between
0.5 and 6 can be expected using the zipHMMlib posterior decoding.

Exploiting repetitions is however possible for posterior decoding if only a
part of the posterior decoding is requested. In that case the running time
becomes $O(M' N^3 + N^2 T' + N^2 (j - i + \log T))$ with $j - i$ being the
length of the requested substring.

zipHMMlib can be extended beyond what has been implemented in this thesis. Below
is listed a number of improvements that can make the library better.
\begin{itemize}
\item In zipHMMlib a parallelized version of the forward algorithm is implemented.
  This is described in detail in \citet{sand2014engineering}. This can be
  implemented similarly for the Viterbi and posterior decoding algorithms. For
  Viterbi the idea is to parallelize equation~\eqref{eq:3} as the max-times
  multiplication can be computed in any order. One processor computes
  equation~\eqref{eq:3} from right to left while the remaining processors
  multiply matrices from the entire equation. Hence, compared to the sequential
  version the first processor does not need to multiply by matrices already
  multiplied by other processors. The first processor will compute
  matrix-vector multiplication while the remaining processors will compute
  matrix-matrix multiplications. Hence, this parallelization scheme will work
  best for small models. Note that $\delta_t$ is not computed for all
  $t \in [1, T]$, since only the first processor makes matrix-vector
  multiplications. Hence, backtracking becomes harder. One solution to this is
  given in \citet{sand2014engineering}, where the missing $\delta_t$ vectors
  are computed and then the path is backtracked. Another idea is to compute
  extra $R$ matrices for each of the matrix-matrix computations. When the
  subset of the $\delta_t$ matrices and the path corresponding to these have
  been computed, the ones missing can be found using the extra $R$ matrices
  using the same technique as in section~\ref{sec:backtracking}.
\item At the current state zipHMMlib compresses the sequences using a single
  compression stopping criterion as discussed in
  section~\ref{sec:compr-stopp-crit}. Since a BLAS framework is used for the
  implementation of the forward and backward algorithms and a naive max-times
  matrix multiplication is used for Viterbi multiple compression criterions
  could be used. Also, the criterion could depend on whether the Viterbi path
  is requested or not. One solution is to compute multiple compressed sequences
  from a single sequence such that the best compression is used for any algorithm.
\item A minor improvement of the indexed posterior decoding algorithm would be
  to save the uncompressed sequence along the compressed sequence. In that case
  no decompression would be needed. However, preliminary experiments suggest
  that this will not improve the running time significantly.
\item As seen in section~\ref{sec:theor-runn-times} the max-times matrix
  multiplication results in cache misses for large models. This problem can be
  solved by either implementing a better max-times multiplication algorithm
  that takes the underlying hardware into account or by storing two copies of
  each matrix with different memory alignment of the data.
\item At the current state the indexed posterior decoding computes the forward
  and backward tables, $\alpha'$ and $\beta'$, and then the indexed posterior
  decoding itself for a pair of indexes $i$ to $j$. If the indexed posterior
  decoding is wanted for multiple pairs of indexes $i_1$ to $j_1$, $i_2$ to
  $j_2$, etc.\ $\alpha'$ and $\beta'$ are recomputed each time. These could be
  saved in memory once computed and reused for each of the pair of indexes.
\item When having an indexed posterior decoding algorithm it would also make
  sense to implement an indexed Viterbi algorithm. This should be faster than
  computing the entire Viterbi path.
\item Finally, \citet{sand2013ziphmmlib} have provided bindings to the Python
  and R programming languages. This has not yet been done for the
  implementations made in this thesis.
\end{itemize}

In most cases the added algorithms to zipHMMlib perform better than the original
algorithms. The algorithms exploiting repetitions should not be used with large models, but this is
also normally preferred as the original algorithms have a running time
proportional to the square of the number of states. In a few cases major
speedups in the order of hundreds are obtained, but in most cases the speedups
are more modest between factors 1 and 10. In some cases the speedup factor is
below 1 which correspond to a slowdown.

As the running time of the algorithms on large data set in some applications
can be measured in hours or days small speedups of a few factors can be very
useful. The original algorithms are very simple and thus it can also hard to
improve on those. In this thesis it is shown that compression of the input can
speed up the algorithms, but that it is only by a small factor in many cases.

%%% Local Variables:
%%% mode: latex
%%% TeX-master: "master.tex"
%%% TeX-command-extra-options: "-shell-escape"
%%% End:
