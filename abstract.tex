\chapter*{Abstract}

This master's thesis describes the theory, implementation and evaluation of
algorithms for hidden Markov model decoding. The classical algorithms used
for decoding is the Viterbi algorithm and the posterior decoding algorithm.
These are reformulated into series of matrix multiplications. By finding
repeated substrings in the observation sequence multiplications can be
avoided and thereby potential for speedup is obtained.

\citet{sand2013ziphmmlib} present zipHMMlib which uses byte-pair encoding to
exploit repetitions. Using and efficient implementation of matrix
multiplication a speedup is gained for the forward algorithm. Theory of
reformulated versions of the Viterbi algorithm, the posterior decoding
algorithm, and a so-called \emph{indexed} posterior decoding algorithm is
presented in this thesis in the context of zipHMMlib and the software library
has been extended to include implementations of these algorithms.

The experiments show that a speedup can be gained in some situations. Speedups
in the order of hundreds or thousands is observed for large repetitive
observation sequences and small models, but in most cases the zipHMMlib version of
the algorithm is between $0.5$ and 10 times faster. The speedup becomes
greater if the sequences are compressed once and used multiple times afterwards
with e.g.\ different models.
%%% Local Variables:
%%% mode: latex
%%% TeX-master: "master"
%%% TeX-command-extra-options: "-shell-escape"
%%% End:
