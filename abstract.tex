\chapter{Abstract}

This master's thesis describes the theory, implementation and evaluation of
algorithms for hidden Markov model decoding. The classical algorithms used
for decoding is the Viterbi algorithm and the posterior decoding algorithm.
These are reformulated into series of matrix multiplications. By finding
repeated substrings in the observation sequence multiplications can be
avoided and thereby a potential speedup is obtained.

\citet{sand2013ziphmmlib} present zipHMMlib that uses byte-pair encoding to
exploit repetitions and by using and efficient implementation of matrix
multiplication a speedup is gained for the forward algorithm. Theory of
reformulated versions of the Viterbi algorithm, the posterior decoding
algorithm, and a so-called \emph{indexed} posterior decoding algorithm is
presented in the context of zipHMMlib and the software library has been
extended to include implementations of these algorithms.

The experiments show that a speedup can be gained in some situations. The
best results are obtained for large repetitive observation sequences and
small models. Also, the overall speedup becomes greater if the sequences are
compressed once and used with different models afterwards.
\fxnote{Is this ok?}
%%% Local Variables:
%%% mode: latex
%%% TeX-master: "master"
%%% TeX-command-extra-options: "-shell-escape"
%%% End:
