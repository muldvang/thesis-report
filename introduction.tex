% chktex-file 44
\chapter{Introduction}

A hidden Markov models is a statistical model that during many years has proven
to be useful in many areas such as speech recognition, character recognition,
face recognition, and various areas of bioinformatics, where it has been
used for e.g.\ gene finding, modeling protein structures and sequence
alignment. Examples of this can be found in \citet{rabiner1989tutorial},
\citet{agazzi1993hidden}, \citet{nefian1998hidden},
\citet{burge1997prediction}, and \citet{eddy1998profile}.

In general, algorithms on hidden Markov models are efficient and can be used
for large data sets in e.g.\ genome wide analysis, but the running time of the
analysis is still measured in hours or days. Since the introduction of next
generation sequencing the amount of available biological data has increased
tremendously, and making faster algorithms for hidden Markov models is thus of
great value.

\citet{lifshits2009speeding} and \citet{sand2013ziphmmlib} use compression of
the input data and minor changes in the algorithms to speed up the
analysis. While \citet{lifshits2009speeding} mainly analyze their method
theoretically, \citet{sand2013ziphmmlib} experimentally proves that the method
has a great potential for performance improvement to the forward algorithm.

The goal of this thesis is to complement the work made by \citet{sand2013ziphmmlib}
by extending the theory and implementation to include the Viterbi and posterior decoding
algorithms and by making experiments validating the performance of the implementations.

\section{Outline}

The thesis is structured as follows.

\begin{description}
\item[Chapter~\ref{cha:background}] provides an overview of the work made by
  \citet{lifshits2009speeding} and \citet{sand2013ziphmmlib} and the differences
  between these.
\item[Chapter~\ref{cha:method}] contains theory of the classical formulations
  of the Viterbi and posterior decoding algorithms along with theoretical
  descriptions and analyses of these algorithms using compression.
\item[Chapter~\ref{cha:implementation}] gives an overview of the implementation
  and instructions on how to use the library.
\item[Chapter~\ref{cha:experiments}] checks that the implementation of the
  algorithms fit the theoretical running time and shows the performance gain
  obtained from the compression.
\item[Chapter~\ref{cha:conclusion}] provides an overview of the developed
  theory, the experiments and the results of these. Ideas on how to
  make the framework better are also given.
\end{description}

%%% Local Variables:
%%% mode: latex
%%% TeX-master: "master"
%%% End:
